Dans cette partie, on propose de factoriser le code d'un duel de dé entre plusieurs joueurs.

\subsection{Premières fonctions}
\begin{UPSTIManipulation}{Test du programme}
    \UPSTIetape Tester le code duel 
    \UPSTIetape Lire l'architecture du programme
\end{UPSTIManipulation}

On propose de découper ce code selon les fonctions suivantes : 
\begin{description}
    \item[\texttt{Lancer\_de}] : Simule le lancement d'un dé
    \begin{description}
        \item[\textbf{Arguments :}] Aucun
        \item[\textbf{Valeur de retour : }] Le résultat du lancer de dé
    \end{description}
    \item[\texttt{Afficher\_gagnant}] : Affiche le gagnant selon les scores donnés en entrée.
    \begin{description}
        \item[\textbf{Arguments :}] les scores des joueurs
        \item[\textbf{Valeur de retour : }] Aucune
        \item[\textbf{Comportement annexe : }] Affiche le gagnant sur la console. 
    \end{description}
\end{description}

\paragraph{} Le programme utilisera alors ces fonctions tout en conservant le même comportement : 

\begin{enumerate}
    \item Deux appels de la fonction \texttt{lancer\_de}
    \item Appel de la fonction \texttt{Afficher\_gagnant}
\end{enumerate}
\begin{UPSTIManipulation}{Factorisation du programme}
    \UPSTIetape Déclarer les fonctions en écrivant les prototypes en début de fichier.
    \UPSTIetape Définir les fonctions correspondantes en fin de fichier
    \UPSTIetape Créer une fonction \texttt{duel} qui prend en entrée le numéro du duel et qui gère un duel.
    \UPSTIetape Modifier la fonction \texttt{main} pour qu'elle appelle les fonctions 
    \UPSTIetape Faire vérifier par l'enseigant.
\end{UPSTIManipulation}

\subsection{Généralisation des fonctions}

On propose à présent d'améliorer les fonctions précédentes pour les rendre plus générale. Dans tous les cas, l'idée est de conserver le même fonctionnement mais de \textbf{généraliser} les fonctions pour les rendre les plus utilies possibles. Les modifications proposées sont les suivantes : 
\paragraph{\text{lancer\_de}} Cette fonction doit maintenant permettre de lancer un dé avec un nombre de face variable. 
On pourrait même proposer le tirage d'un nombre compris entre deux bornes données en paramètres.


\begin{UPSTIManipulation}{Généralisation des fonctions}
    \UPSTIetape Modifier les fonctions pour obtenir les généralisations présentées ci-dessus.
    \UPSTIetape Faire vérifier par l'enseignant.
\end{UPSTIManipulation}

\begin{UPSTIManipulation}{Bonus : Généralisation de la fonction Afficher\_gagnant}
    \UPSTIetape Modifier la fonction \texttt{Afficher\_gagnant} pour qu'elle puisse gérer un nombre variable de joueurs.
    \UPSTIetape Faire vérifier par l'enseignant.
\end{UPSTIManipulation}