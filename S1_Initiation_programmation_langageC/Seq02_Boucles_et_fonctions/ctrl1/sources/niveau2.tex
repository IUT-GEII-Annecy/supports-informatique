% ==================== NIVEAU 2 — VARIABLES / TYPES (10) ====================

\element{niveau2}{
  \begin{question}{n1-intervalle}
    Comment écrire la condition \(5 \le x \le 10\) (inclus) :
    \begin{multicols}{4}
      \begin{reponses}
        \bonne{\lstinline[language=C]|x >= 5 && x <= 10|}
        \mauvaise{\lstinline[language=C]|x > 5 || x < 10|}
        \mauvaise{\lstinline[language=C]|x > 5 && x < 10|}
        \mauvaise{\lstinline[language=C]|x <= 5 && x >= 10|}
      \end{reponses}
    \end{multicols}

  \end{question}
}

\element{niveau2}{
  \begin{question}{n2-priorite-ops}
    Quelle est la valeur finale de \lstinline|r| ?

    \lstinline[language=c]|int r = 10 - 2 * 3 + 8 / 2;|

    \begin{multicols}{5}\begin{reponses}
      \bonne{8}
      \mauvaise{16}
      \mauvaise{-1}
      \mauvaise{4}
      \mauvaise{10}
      \mauvaise{0}
    \end{reponses}\end{multicols}
  \end{question}
}

\element{niveau2}{
  \begin{question}{n2-types-cast}
    Quelle est la valeur de ces expressions ?
    \lstinputlisting[language=c]{sources/codes/niveau2/valeur-1.c}
    \begin{multicols}{5}\begin{reponses}
      \bonne{2.6}
      \mauvaise{2}
      \mauvaise{2.5}
      \mauvaise{13/5}
      \mauvaise{3}
      \mauvaise{Erreur}
    \end{reponses}\end{multicols}
  \end{question}
}

\element{niveau1-open}{
  \begin{question}{n2-open-declaration}
    Déclarer et initialiser des variables adaptées pour : masse d'un objet (kg, décimales), nombre de pièces (entier), code article (texte).
    \evaluationProfLignes{1}{3}
  \end{question}
}

\element{niveau2}{
  \begin{question}{n2-open-trace}
    Prédire les valeurs finales de \lstinline|x|, \lstinline|y|, \lstinline|z|.
    \lstinputlisting[language=c]{sources/codes/niveau2/calculs-1.c}
    \evaluationProfLignes{2}{2}
  \end{question}
}

\element{niveau2}{
  \begin{question}{n2-conv-mix}
    Quel est le résultat de \lstinline|(int)(9.0/2*10)| ?
    \begin{multicols}{6}
      \begin{reponses}
        \bonne{45}
        \mauvaise{45.0}
        \mauvaise{2}
        \mauvaise{2.25}
        \mauvaise{9}
        \mauvaise{Erreur}
      \end{reponses}
    \end{multicols}

  \end{question}
}

\element{niveau2}{
  \begin{question}{n2-open-init}
    Écrire une initialisation pour : booléen \texttt{isRegistered}, réel \texttt{moyenne}, entier \texttt{nbAbsences}.
    \evaluationProfLignes{2}{2}
  \end{question}
}

\element{niveau2}{
  \begin{question}{n2-affichage-melange}
    Que va afficher ce code ?
    \lstinputlisting[language=c]{sources/codes/niveau2/predire-1.c}
    \begin{multicols}{5}\begin{reponses}
      \bonne{7  4.50}
      \mauvaise{7.00  4.50}
      \mauvaise{7 4}
      \mauvaise{7.00  4}
      \mauvaise{9 2.25}
      \mauvaise{7 9/4}
      \mauvaise{7 5}
      \lastchoices
      \mauvaise{Erreur}
    \end{reponses}\end{multicols}
  \end{question}
}

\element{niveau1-open}{
  \begin{question}{n2-open-type-justif}
    Pour : tension électrique (12.5 V), nombre d’interrupteurs (3), état marche/arrêt, proposer un type et justifier.
    \evaluationProfLignes{3}{3}
  \end{question}
}

\element{niveau2}{
  \begin{question}{n2-modulo}
    Que vaut \lstinline|28 \% 5| ?
    \begin{multicols}{5}\begin{reponses}
      \bonne{3}
      \mauvaise{5}
      \mauvaise{0}
      \mauvaise{2}
      \mauvaise{4}
      \mauvaise{23}
      \mauvaise{1}
      \lastchoices
      \mauvaise{Erreur}
    \end{reponses}\end{multicols}
  \end{question}
}

\element{bonus}{
  \begin{question}{n2-open-erreurs}
    Pourquoi la somme répétée de 0.1 en \lstinline|float| peut-elle ne pas donner exactement 1.0 ?
    Expliquer en deux phrases.
    \evaluationProfLignes{2}{2}
  \end{question}
}

% ==================== NIVEAU 2 — CONDITIONS (10) ====================

\element{niveau2}{
  \begin{question}{n2-cond-majorite}
    On veut distinguer : Les enfants (moins de 12 ans), des ados et des adultes (plus de 18 ans).
    Quelle structure est correcte ?
    \begin{reponses}
      \bonne{\lstinline[language=C]|if(age<12)... else if(age<18)... else...|}
      \mauvaise{\lstinline[language=C]|if(age<12)... if(age<18)... else...|}
      \mauvaise{\lstinline[language=C]|if(age<18)... else if(age<12)... else...|}
      \lastchoices
      \mauvaise{Impossible en C}\bareme{-1}
    \end{reponses}
  \end{question}
}

\element{niveau2}{
  \begin{question}{n2-cond-intervalle}
    Condition correcte pour tester \texttt{(0 < t < 100)} :
    \begin{multicols}{5}\begin{reponses}
      \bonne{\lstinline[language=C]{t>0 && t<100}}
      \mauvaise{\lstinline[language=C]{t>0 || t<100}}
      \mauvaise{\lstinline[language=C]|t>=0 && t<=100|}
      \mauvaise{\lstinline[language=C]|t=0 && t=100|}
      \mauvaise{\lstinline[language=C]|t<0 && t>100|}
      \lastchoices
      \mauvaise{Impossible}
    \end{reponses}\end{multicols}
    \end{question}
}

\element{niveau2}{
  \begin{question}{n2-open-acces}
    On veut : Accès si \texttt{isRegistered} et \texttt{hasBadge}, ou si \texttt{isAdmin}.
    Écrire la condition en C.
    \evaluationProfLignes{2}{2}
  \end{question}
}

\element{niveau2}{
  \begin{question}{n2-bool-eval1}
    Avec a=3, b=7, c=7, que vaut \lstinline|(a<b)&&(b==c)| ?
    \begin{multicols}{5}\begin{reponses}
      \bonne{true}
      \mauvaise{false}
      \mauvaise{Erreur}
      \mauvaise{0}
      \mauvaise{1}\bareme{0.5}
      \mauvaise{Inconnu}
    \end{reponses}\end{multicols}
  \end{question}
}

\element{niveau2}{
  \begin{question}{n2-bool-eval2}
    Avec a=3, b=7, c=7, que vaut \lstinline|(a>=b)||(c!=7)| ?
    \begin{multicols}{5}\begin{reponses}
      \bonne{false}
      \mauvaise{true}
      \mauvaise{Erreur}
      \mauvaise{0}\bareme{0.5}
      \mauvaise{1}
      \mauvaise{Inconnu}
    \end{reponses}\end{multicols}
  \end{question}
}

\element{niveau2}{
  \begin{question}{n2-open-traduction}
    Traduire en condition C :
    \og On accorde une réduction si l'étudiant a une moyenne supérieure ou égale à 14 et est assidu \fg{}.
    \evaluationProfLignes{1}{2}
  \end{question}
}

\element{niveau2}{
  \begin{question}{n2-cond-syntaxe}
    Laquelle est une écriture correcte ?
    \begin{multicols}{5}\begin{reponses}
      \bonne{\lstinline[language=C]|if(x>=0 && x<=10) printf("OK");|}
      \mauvaise{\lstinline[language=C]|if(x>=0 && x<=10) { printf("OK") }|}
      \mauvaise{\lstinline[language=C]|if x>=0 && x<=10 then printf("OK");|}
      \mauvaise{\lstinline[language=C]|if(x>=0 && x<=10); printf("OK");|}
      \mauvaise{\lstinline[language=C]|if(x>=0) and (x<=10) printf("OK");|}
      \mauvaise{Aucune}
    \end{reponses}\end{multicols}
  \end{question}
}

\element{niveau2}{
  \begin{question}{n2-open-cinema}
    Rédiger une condition pour afficher le tarif :
    <12 ans = 4€, 12–25 ans = 6€, >25 ans = 9€.
    \evaluationProfLignes{1}{3}
  \end{question}
}

\element{niveau2}{
  \begin{question}{n2-bool-eval3}
    Avec a=4, b=7, que vaut \lstinline|!((a\%2)\&\&(b\%2))| ?
    \begin{multicols}{5}\begin{reponses}
      \bonne{vrai}
      \mauvaise{faux}
      \mauvaise{Erreur}
      \mauvaise{0}
      \mauvaise{1}/bareme{0.5}
      \mauvaise{2}
    \end{reponses}\end{multicols}
  \end{question}
}

