\section{Projet 11 — Mastermind -- Suite de couleurs}
\setcounter{UPSTInumeroCahier des charges}{0}

\subsection*{Objectif du jeu}
Deviner une suite cachée de 4 couleurs codées (0..3) avec indication des bien/mal placées.

\subsection*{Règles du jeu}
\begin{itemize}
  \item Le programme tire au hasard une \textbf{suite de 4 couleurs} parmi Rouge, Vert, Bleu, Noir (\textbf{avec répétitions possibles}).
  \item Codage : \texttt{0}=Noir, \texttt{1}=Rouge, \texttt{2}=Vert, \texttt{3}=Bleu.
  \item Le joueur saisit un \textbf{entier à 4 chiffres}; le programme le \textbf{décompose} en tableau.
  \item Le programme indique \textbf{nombre de bien placées} et \textbf{nombre de mal placées}. Score = tentatives × temps / 10.
\end{itemize}

\subsection*{Cahier des charges}
\begin{UPSTICahierDesCharges}{1}
\begin{itemize}
  \item Tirer la suite et \textbf{mémoriser} le code (tableau 4 cases); \textbf{ne pas l’afficher}.
  \item Le joueur saisit la proposition (entier 4 chiffres), \textbf{décomposer} en tableau, comparer et afficher les compteurs.
  \item Boucler jusqu’à la bonne réponse.
\end{itemize}
\end{UPSTICahierDesCharges}

\begin{UPSTICahierDesCharges}{2}
\begin{itemize}
  \item Gérer les répétitions correctement (bien/mal placées).
  \item Ajouter \textbf{chrono} et calcul du score; récapitulatif.
  \item Valider la saisie.
\end{itemize}
\end{UPSTICahierDesCharges}

\begin{UPSTICahierDesCharges}{3}
\begin{itemize}
  \item Mode \textbf{2 joueurs} en alternance; afficher les tentatives de chacun.
  \item Présentation (affichage de carrés ANSI) et rejouabilité.
  \item Factoriser en fonctions.
\end{itemize}
\end{UPSTICahierDesCharges}

\subsection*{Squelette de départ}
\lstinputlisting[language=C]{suitecouleurs_squelette.c}
