\section{Projet 10 — Le pendu}
\setcounter{UPSTInumeroCahier des charges}{0}

\subsection*{Objectif du jeu}
Faire deviner un mot lettre par lettre avec 7 erreurs autorisées, en affichant la progression.

\subsection*{Règles du jeu}
\begin{itemize}
  \item Un utilisateur saisit un \textbf{mot} (minuscule, sans accents ni espaces).
  \item Un joueur propose des lettres; le programme affiche le mot avec \_ pour les lettres non trouvées.
  \item À chaque lettre absente, \textbf{+1 erreur}. La partie s’arrête à 7 erreurs ou si le mot est trouvé.
  \item Variante 2 joueurs : inversion des rôles, 3 manches.
\end{itemize}

\subsection*{Cahier des charges}
\begin{UPSTICahierDesCharges}{1}
\begin{itemize}
  \item Saisir le mot \textbf{secret}, l’afficher \textbf{caractère par caractère} pour vérification, puis masquer.
  \item Boucle de jeu : saisie d’une lettre, mise à jour de l’affichage, \textbf{comptage des erreurs}.
  \item Condition de victoire/défaite.
\end{itemize}
\end{UPSTICahierDesCharges}

\begin{UPSTICahierDesCharges}{2}
\begin{itemize}
  \item Validation des saisies, gestion des lettres déjà proposées.
  \item Affichage du temps de jeu.
\end{itemize}
\end{UPSTICahierDesCharges}

\begin{UPSTICahierDesCharges}{3}
\begin{itemize}
  \item Présentation soignée -> Dessin ?
  \item Mode 2 joueurs avec alternance des rôles sur 3 manches.
  \item Option : dictionnaire de mots.
\end{itemize}
\end{UPSTICahierDesCharges}

\subsection*{Squelette de départ}
\lstinputlisting[language=C]{pendu_squelette.c}
