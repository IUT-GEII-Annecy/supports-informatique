Vous pouvez télécharger un squelette sur le lien suivant : 

\begin{lstlisting}[language=bash]
wget https://github.com/IUT-GEII-Annecy/squelettes/releases/download/branch-2025/tp6.zip
\end{lstlisting}
Dans cette première partie, vous allez découvrir les structures en manipulant des points dans un plan 2D.

\subsection{Déclaration et utilisation basique}

\begin{UPSTIManipulation}{Déclarer et afficher un point}
  \textbf{Objectif :} Créer une structure Point et l'utiliser.

  \textbf{Cahier des charges :}
  \begin{itemize}
    \item Déclarer une structure \texttt{Point} avec deux champs \texttt{float x} et \texttt{float y}
    \item Dans le \texttt{main}, créer une variable \texttt{p1} de type \texttt{Point}
    \item Demander à l'utilisateur les coordonnées x et y
    \item Afficher le point sous la forme : \texttt{Point: (x, y)}
  \end{itemize}

  \textbf{Exemple d'exécution :}
  \begin{lstlisting}[style=console]
Entrez x: 3.5
Entrez y: 2.8
Point: (3.5, 2.8)
  \end{lstlisting}

  \tcblower
  \begin{lstlisting}[language=bash]
check50 IUT-GEII-Annecy/exercices/2025/info1/tp_structures/point_simple
  \end{lstlisting}
\end{UPSTIManipulation}

\begin{UPSTIManipulation}{Fonction de création de point}

  \textbf{Objectif :} Créer une fonction qui retourne une structure.

  \textbf{Cahier des charges :}
  \begin{itemize}
    \item Écrire une fonction \texttt{Point creer\_point(float x, float y)} qui crée et retourne un \texttt{Point}
    \item Écrire une fonction \texttt{void afficher\_point(Point p)} qui affiche un point sous la forme \texttt{(x, y)}
    \item Dans le \texttt{main}, utiliser ces fonctions pour créer et afficher 3 points
  \end{itemize}

%   \tcblower
%   \begin{lstlisting}[language=bash]
% check50 IUT-GEII-Annecy/exercices/2025/info1/tp_structures/creer_point
%   \end{lstlisting}
\end{UPSTIManipulation}

\begin{UPSTIManipulation}{Calculer la distance à l'origine}

  \textbf{Contexte :} On souhaite calculer la distance entre un point et l'origine (0, 0).

  \textbf{Formule :} $d = \sqrt{x^2 + y^2}$ (utiliser \texttt{sqrt} de \texttt{math.h})

  \textbf{Cahier des charges :}
  \begin{itemize}
    \item Reprendre la structure \texttt{Point}
    \item Écrire une fonction \texttt{float distance\_origine(Point p)} qui calcule et retourne la distance
    \item Dans le \texttt{main}, créer un point, demander ses coordonnées et afficher sa distance à l'origine
    \item Arrondir l'affichage à 2 décimales
  \end{itemize}

  \textbf{Note :} Pour compiler avec \texttt{math.h}, ajouter \texttt{-lm} : \texttt{gcc fichier.c -lm}

  \tcblower
  \begin{lstlisting}[language=bash]
check50 IUT-GEII-Annecy/exercices/2025/info1/tp_structures/distance_origine
  \end{lstlisting}
\end{UPSTIManipulation}

\subsection{Fonctions sur les structures}

\begin{UPSTIManipulation}{Distance entre deux points}

  \textbf{Contexte :} Calculer la distance entre deux points quelconques.

  \textbf{Formule :} $d = \sqrt{(x_2-x_1)^2 + (y_2-y_1)^2}$

  \textbf{Cahier des charges :}
  \begin{itemize}
    \item Écrire une fonction \texttt{float distance(Point p1, Point p2)}
    \item Dans le \texttt{main}, demander les coordonnées de deux points
    \item Afficher la distance entre ces deux points (2 décimales)
  \end{itemize}

  \textbf{Exemple :}
  \begin{lstlisting}[style=console]
Point 1 - x: 0
Point 1 - y: 0
Point 2 - x: 3
Point 2 - y: 4
Distance: 5.00
  \end{lstlisting}

  \tcblower
  \begin{lstlisting}[language=bash]
check50 IUT-GEII-Annecy/exercices/2025/info1/tp_structures/distance_deux_points
  \end{lstlisting}
\end{UPSTIManipulation}


