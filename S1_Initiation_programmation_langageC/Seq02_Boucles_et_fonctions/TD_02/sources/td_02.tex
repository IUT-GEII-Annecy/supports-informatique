\section{Exercices complémentaires}


\begin{UPSTIexercice}{Rectangle vide}
    \begin{minipage}{0.8\textwidth}
  \UPSTIquestion{Ecrire un programme qui affiche un rectangle vide de largeur `w` et de hauteur `h` donnés par l'utilisateur. Par exemple, pour `w=6` et `h=4`, le programme affichera le dessin ci-contre.}
\end{minipage}%
\hfill
    \begin{minipage}{0.1\textwidth}
\begin{verbatim}
******
*    *
*    *
******\end{verbatim}
\end{minipage}
\end{UPSTIexercice}


\begin{UPSTIexercice}{Triangle de Pascal}
    \begin{minipage}
{0.8\textwidth}
 Le triangle de Pascal est une disposition triangulaire des coefficients binomiaux. Chaque nombre est la somme des deux nombres directement au-dessus de lui dans le triangle.
      \UPSTIquestion{Ecrire un programme qui affiche les `n` premières lignes du triangle de Pascal, où `n` est un entier donné par l'utilisateur. Par exemple, pour `n=5`, le programme affichera le dessin ci-contre.
}\end{minipage}
\hfill
\begin{minipage}{0.1\textwidth}
\begin{verbatim}
    1
   1 1
  1 2 1
 1 3 3 1
1 4 6 4 1\end{verbatim}
\end{minipage}
\end{UPSTIexercice}

\begin{UPSTIexercice}{Prédire l'affichage}
  \UPSTIquestion{Prédire l'affichage généré par le code suivant :
\begin{lstlisting}[language=C]
#include <stdio.h>
int main() {
    int n = 5;
    for (int i = 1; i <= n; i++) {
        for (int j = 1; j <= i; j++) {
            printf("*");
        }
        printf("\n");
    }
    return 0;
}
\end{lstlisting}}
  \UPSTIquestion{Prédire l'affichage généré par le code suivant :
\begin{lstlisting}[language=C]
#include <stdio.h>
int main() {
    int n = 5;
    for (int i = n; i >= 1; i--) {
        for (int j = 1; j <= i; j++) {
            printf("*");
        }
        printf("\n");
    }
    return 0;
}
\end{lstlisting}}
  \UPSTIquestion{Prédire l'affichage généré par le code suivant : 
\begin{lstlisting}[language=C]
#include <stdio.h>
int main() {
    int n = 5;
    for (int i = 1; i <= n; i++) {
        for (int j = 1; j <= n - i; j++) {
            printf(" ");
        }
        for (int k = 1; k <= i; k++) {
            printf("*");
        }
        printf("\n");
    }
    return 0;
}
\end{lstlisting}}
\end{UPSTIexercice}

\begin{UPSTIprofOnlyEnv}
\begin{UPSTIcorrectionP}{Prédire l'affichage}
    Voici les affichages attendus pour les trois exercices de prédiction :
    \begin{itemize}
        \item Premier exercice :
        \begin{verbatim}
*
**
***
****
*****
        \end{verbatim}
        \item Deuxième exercice :
        \begin{verbatim}*****
*****
****
***
**
*
        \end{verbatim}
        \item Troisième exercice :
        \begin{verbatim}
    *
   **
  ***
 ****
*****
        \end{verbatim}
    \end{itemize}
\end{UPSTIprofOnlyEnv}

\section{Très difficile}
\begin{UPSTIexercice}{Disque et cercle}
  \UPSTIquestion{Ecrire un programme qui affiche un disque plein et un cercle vide de rayon `r` donnés par l'utilisateur. Par exemple, pour `r=5`, le programme affichera :
}\begin{verbatim}
  *****
 *******
*********
*********
*********
*********
 *******
  *****\end{verbatim}
  
\end{UPSTIexercice}

\begin{UPSTIprofOnlyEnv}
  \begin{UPSTIcorrectionP}{Disque et cercle}
    Voici un exemple de solution pour afficher un cercle vide. L'affichage d'un disque plein est laissé en exercice.
    \begin{lstlisting}[language=C]
#include <stdio.h>
#include <math.h>
#define PI 3.14159265358979323846
int main() {
    int r;
    printf("Entrez le rayon du cercle : ");
    scanf("%d", &r);

    for (int y = -r; y <= r; y++) {
        for (int x = -r; x <= r; x++) {
            if (fabs(sqrt(x*x + y*y) - r) < 0.5) {
                printf("*");
            } else {
                printf(" ");
            }
        }
        printf("\n");
    }
    return 0;
}
    \end{lstlisting}
  \end{UPSTIcorrectionP}
\end{UPSTIprofOnlyEnv}