

\begin{UPSTIinfor}{Concevoir une fonction}
  Pour être bien conçue, une fonction doit respecter les critères suivants :
  \begin{itemize}
    \item ne doit faire qu'une seule tâche (principe de responsabilité unique) ;
          \begin{itemize}
            \item Rend le code explicite,
            \item Facilite la maintenance,
            \item Facilite les tests unitaires.
            \item Favorise la réutilisabilité.
          \end{itemize}
    \item être la plus générique possible :
          \begin{itemize}
            \item Cela signifie qu'elle doit être la plus indépendante possible du contexte dans lequel elle est utilisée.
          \end{itemize}
    \item avoir un nom explicite qui reflète son rôle ;
    \item avoir des paramètres clairs et pertinents ;
    \item avoir un type de retour approprié.
  \end{itemize}
\end{UPSTIinfor}

\begin{UPSTIinfor}{Point méthodologique}
  Avant de se lancer dans le codage, il est important de bien réfléchir à la conception des fonctions. Voici quelques étapes à suivre :
  \begin{itemize}
    \item Identifier les différentes tâches dans le code existant ;
    \item Regrouper les tâches similaires ou liées ;
    \item Généraliser les tâches identifiées
    \item Définir les fonctions en fonction des tâches identifiées ;
    \item Déterminer les paramètres nécessaires pour chaque fonction ;
    \item Définir le type de retour pour chaque fonction.
  \end{itemize}
\end{UPSTIinfor}

\pagebreak