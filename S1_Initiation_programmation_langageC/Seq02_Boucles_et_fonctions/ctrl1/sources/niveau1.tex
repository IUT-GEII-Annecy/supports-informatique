% ==================== NIVEAU 1 — VARIABLES / TYPES (10) ====================

\element{niveau1}{
  \begin{question}{n1-types-annee-naissance}
    Quel type est cohérent pour représenter l'année de naissance d'un étudiant ?
    \begin{multicols}{5}
      \begin{reponses}
        \bonne{\lstinline[language=c]|int|}
        \mauvaise{\lstinline[language=c]|float|}
        \mauvaise{\lstinline[language=c]|double|}
        \mauvaise{\lstinline[language=c]|string|}
        \mauvaise{\lstinline[language=c]|bool|}
      \end{reponses}
    \end{multicols}
  \end{question}
}

\element{niveau1}{
  \begin{question}{n1-types-initiale}
    Quel type convient pour stocker une \textbf{initiale} (une seule lettre) ?
    \begin{multicols}{5}
      \begin{reponses}
        \bonne{\lstinline[language=c]|char|}
        \mauvaise{\lstinline[language=c]|int|}
        \mauvaise{\lstinline[language=c]|float|}
        \mauvaise{\lstinline[language=c]|double|}
        \mauvaise{\lstinline[language=c]|string|}
      \end{reponses}
    \end{multicols}
  \end{question}
}

\element{niveau1}{
  \begin{question}{n1-types-temperature}
    Pour une température mesurée avec décimales, le type le plus adapté est :
    \begin{multicols}{5}
      \begin{reponses}
        \mauvaise{\lstinline[language=c]|char|}
        \mauvaise{\lstinline[language=c]|int|}
        \bonne{\lstinline[language=c]|float|}
        \mauvaise{\lstinline[language=c]|double|}\bareme{1}
        \mauvaise{\lstinline[language=c]|string|}
      \end{reponses}
    \end{multicols}
  \end{question}
}

\element{niveau1}{
  \begin{question}{n1-types-salle}
    Le \textbf{numéro de salle} (par ex. 105) devrait être stocké en :
    \begin{multicols}{5}
      \begin{reponses}
        \bonne{\lstinline[language=c]|int|}
        \mauvaise{\lstinline[language=c]|float|}
        \mauvaise{\lstinline[language=c]|char|}
        \mauvaise{\lstinline[language=c]|double|}
        \mauvaise{\lstinline[language=c]|float|}
      \end{reponses}
    \end{multicols}
  \end{question}
}

\element{niveau1}{
  \begin{question}{n1-types-prix}
    Un \textbf{prix} (ex. 3.20€) doit être stocké idéalement en :
    \begin{multicols}{5}
      \begin{reponses}
        \bonne{\lstinline[language=c]|double|}
        \mauvaise{\lstinline[language=c]|int|}
        \mauvaise{\lstinline[language=c]|char|}
        \mauvaise{\lstinline[language=c]|bool|}
        \mauvaise{\lstinline[language=c]|float|}\bareme{0.5}
      \end{reponses}
    \end{multicols}
  \end{question}
}

\element{niveau1-open}{
  \begin{question}{n1-open-declarations}
    Écrire la \textbf{déclaration} (sans initialisation) de variables pour : prénom, âge, taille (en m).
    \evaluationProfLignes{3}{3}
  \end{question}
}


\element{niveau1}{
  \begin{question}{n1-trace-affectations}
    Que valent \lstinline|x|, \lstinline|y|, \lstinline|z| après ce code ?
    \lstinputlisting[language=c]{sources/codes/affectation-1.c}
    \begin{multicols}{5}
      \begin{reponses}
        \bonne{\lstinline|x=13, y=6, z=7|}
        \mauvaise{\lstinline|x=7, y=6, z=13|}
        \mauvaise{\lstinline|x=14, y=5, z=7|}
        \mauvaise{\lstinline|x=14, y=6, z=7|}
        \mauvaise{\lstinline|x=13, y=7, z=7|}
        \mauvaise{\lstinline|x=7, y=7, z=13|}
        \mauvaise{\lstinline|x=14, y=5, z=13|}
        \mauvaise{\lstinline|x=15, y=6, z=13|}
      \end{reponses}
    \end{multicols}

  \end{question}
}

\element{niveau1}{
  \begin{question}{n1-prediction-division}
    Prédire l'affichage :
    \lstinputlisting[language=c]{sources/codes/division.c}
    \begin{multicols}{5}
      \begin{reponses}
        \bonne{\lstinline|C:3| \\ \lstinline|D:3.500000|}
        \mauvaise{\lstinline|C:3| \\ \lstinline|D:3.000000|}
        \mauvaise{\lstinline|C:3.500000| \\ \lstinline|D:3|}
        \mauvaise{\lstinline|C:3| \\ \lstinline|D:3.5|}
        \mauvaise{\lstinline|C:3.000000| \\ \lstinline|D:3.500000|}
      \end{reponses}
    \end{multicols}
  \end{question}
}

% ==================== NIVEAU 1 — CONDITIONS (10) ====================

\element{niveau1}{
  \begin{question}{n1-if-majeur}
    Que doit afficher ce code si \lstinline|age=16| ?
    \lstinputlisting[language=c]{sources/codes/bonjour.c}
    \begin{multicols}{5}
      \begin{reponses}
        \bonne{\lstinline|Au revoir|\\}
        \mauvaise{\lstinline|Bienvenue|\\}
        \mauvaise{\lstinline|Bienvenue|\\\lstinline|Au revoir|}
        \mauvaise{Rien\\}
        \mauvaise{Erreur\\}
      \end{reponses}
    \end{multicols}
  \end{question}
}

\element{niveau1}{
  \begin{question}{n1-parite}
    Quelle condition teste que \texttt{n} est \textbf{pair} ?
    \begin{multicols}{5}
    \begin{reponses}
      \bonne{\lstinline[language=C]|n \% 2 == 0|}
      \mauvaise{\lstinline[language=C]|n / 2 == 0|}
      \mauvaise{\lstinline[language=C]|n == 2|}
      \mauvaise{\lstinline[language=C]|n \% 2 == 1|}
      \mauvaise{\lstinline[language=C]|n / 2 == 1|}
    \end{reponses}
  \end{multicols}
  \end{question}
}

\element{niveau1-open}{
  \begin{question}{n1-open-comparaison}
    Écrire une suite \lstinline|if/else if/else| qui affiche le plus grand entre \lstinline|a| et \lstinline|b|, ou \og Égaux \fg{}.
    \evaluationProfLignes{3}{3}
  \end{question}
}

\element{niveau1}{
  \begin{question}{n1-prediction-simple}
    Sortie du programme ?
    \lstinputlisting[language=c]{sources/codes/prediction-simple-1.c}
    \begin{multicols}{5}
      \begin{reponses}
        \bonne{\lstinline|<=5|}
        \mauvaise{\lstinline|>5|}
        \mauvaise{Rien}
        \mauvaise{Erreur}
      \end{reponses}
    \end{multicols}
  \end{question}
}

\element{niveau1-if-open}{
  \begin{question}{n1-open-nbmystere}
    On a \lstinline|int mystere=7;| et l'utilisateur propose \lstinline|p|.
    Ecrire un code qui répond \texttt{Bravo !} si l'utilisateur a trouvé. Sinon, il indiquera \texttt{Plus grand} ou \texttt{Plus petit} afin de l'aider s'il n'a pas trouvé. 
    \evaluationProfLignes{3}{4}
  \end{question}
}

\element{niveau1}{
  \begin{question}{n1-if-syntaxe}
    Quelle écriture est \textbf{correcte} ?
    \begin{multicols}{5}
      \begin{reponses}
        \bonne{\lstinline[language=C]|if (age>=18) { printf("Majeur"); } else { printf("Mineur"); }|}
        \mauvaise{\lstinline[language=C]|if (age>=18) printf("Majeur") else printf("Mineur");|}
        \mauvaise{\lstinline[language=C]|if age>=18 then printf("Majeur");|}
        \mauvaise{\lstinline[language=C]|if (age>=18) { printf("Majeur") } else printf("Mineur")|}
      \end{reponses}
    \end{multicols}
  \end{question}
}

\element{niveau1}{
  \begin{question}{n1-prediction-multi}
    Prédire la sortie de ce programme :
    \lstinputlisting[language=c]{sources/codes/prediction-simple-2.c}
    \begin{multicols}{5}
      \begin{reponses}
        \bonne{\lstinline|E|}
        \mauvaise{\lstinline|A|}
        \mauvaise{\lstinline|B|}
        \mauvaise{Rien}
      \end{reponses}
    \end{multicols}

  \end{question}
}

\element{niveau1-open}{
  \begin{question}{n1-open-cond-rediger}
    Rédiger une condition qui affiche \og Accès \fg{} si \lstinline|badge==1|, sinon \og Refus \fg{}.
    \evaluationProfLignes{3}{1}
  \end{question}
}
