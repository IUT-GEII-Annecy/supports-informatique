
\begin{UPSTIexercice}{Identifier les fonctions}
Voici un extrait de programme :
\begin{lstlisting}[language=C]
#include <stdio.h>

int somme(int a, int b);

int main(void) {
    int x = 4;
    int y = 5;
    int z = somme(x, y);
    printf("%d\n", z);
}

int somme(int a, int b) {
    return a + b;
}
\end{lstlisting}

\UPSTIquestion{Donner le nom de la fonction définie par le programme.}
\UPSTIquestion{Quels sont ses paramètres ? Quelle est sa valeur de retour ?}
\UPSTIquestion{Où la fonction est-elle appelée ? Que va afficher le programme ?}
\end{UPSTIexercice}


\begin{UPSTIexercice}{Compléter une fonction}
On souhaite écrire une fonction \texttt{difference(a,b)} qui renvoie la différence absolue entre deux nombres entiers.
\UPSTIquestion{Dessiner cette fonction sous la forme d'une boite avec entrées et sorties}
\UPSTIquestion{Donner le prototype de la fonction}
\UPSTIquestion{Donner le corps de la fonction}
\end{UPSTIexercice}


\begin{UPSTIexercice}{Calculer et afficher}
\UPSTIquestion{Écrire une fonction \texttt{int carre(int n)} qui renvoie le carré d’un entier.}
\UPSTIquestion{Ecrire une fonction \texttt{void afficher\_carre(int n)} qui affiche la phrase :
\begin{center}
\texttt{Le carré de n est ...}
\end{center}}

\UPSTIquestion{Comment la deuxième fonction peut-elle utiliser la première ?}
\UPSTIquestion{Pourquoi est-il utile de séparer les deux fonctions ?}
\end{UPSTIexercice}


\begin{UPSTIexercice}{Fonctions avec plusieurs appels}

\UPSTIquestion{Ecrire une fonction minimum qui renvoie le minimum entre deux nombres.}

On veut une fonction \texttt{int minimum3(int a, int b, int c)} qui renvoie le plus grand de trois entiers, en réutilisant la fonction \texttt{minimum}.

\UPSTIquestion{Écrire la fonction \texttt{minimum3}.}
\UPSTIquestion{Quel intérêt y a-t-il à réutiliser une fonction déjà écrite ?}
\UPSTIquestion{Comment tester cette fonction rapidement dans le \texttt{main} ?}
\end{UPSTIexercice}


\begin{UPSTIexercice}{Heure en seconde}
    \UPSTIquestion{Donner le prototype et le corps d'une fonction qui prend en paramètre le nombre d'\texttt{heures}, \texttt{minutes} et \texttt{secondes} et qui donne en sortie le nombre de secondes écoulées depuis le début de la journée.}
\end{UPSTIexercice}

\begin{UPSTIexercice}{Fonctions booléennes}
On définit une fonction \texttt{est\_pair} qui teste si un nombre est pair :

\UPSTIquestion{Ecrire le prototype puis le code de la fonction.}
\UPSTIquestion{Proposer une autre fonction \texttt{est\_multiple\_de(int n, int m)}.}
\end{UPSTIexercice}



\begin{UPSTIexercice}{S'appeler soi-même}
    Soit la fonction suivante : 
    \begin{lstlisting}[language=c]
        int partez(int n){
            if (n > 0){
                printf("%i, ",n);
                partez(n-1);
            }
            else{
                printf("GO !");
            }
        }
    \end{lstlisting}  
    
    \UPSTIquestion{Prédire l'affichage produit par les appels suivants}
    \begin{multicols}{4}
    \begin{itemize}
        \item \lstinline[language=c]|partez(0)|
        \item \lstinline[language=c]|partez(1)|
        \item \lstinline[language=c]|partez(3)|
        \item \lstinline[language=c]|partez(5)|
    \end{itemize}
    \end{multicols}
    

    Une fonction qui s'appelle elle-même est appelée une \textbf{fonction récursive.}
\end{UPSTIexercice}



\begin{UPSTIexercice}{Fonctions normales ou récursives ?}
On souhaite calculer la factorielle d’un entier positif $n$.

\UPSTIquestion{Rappeler la définition mathématique de la factorielle.}
\UPSTIquestion{Donner une fonction qui calcule la factorielle d'un nombre donné en paramètres.}
\UPSTIquestion{\textbf{Difficile : }Sans utiliser de boucle, écrire une fonction qui calcule la factorielle en s'appelant elle-même.}
\end{UPSTIexercice}

\begin{UPSTIexercice}{Nombre binaire}
    \UPSTIquestion{Ecrire une fonction qui convertie un nombre décimal en binaire avec la méthode de la division.}
    \UPSTIquestion{Même question avec une fonction récursive.}
\end{UPSTIexercice}

