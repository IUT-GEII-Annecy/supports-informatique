Ces exercices bonus permettent d'approfondir les structures avec des applications plus complexes.

\subsection{Exercices bonus}

\begin{UPSTIManipulation}{Gestion de stock — Articles les plus chers}
  \textbf{Contexte :} Un magasin souhaite identifier ses 3 articles les plus chers en stock.

  \textbf{Structure à utiliser :}
  \begin{lstlisting}[language=C]
typedef struct {
    int code;            // Code article
    char nom[50];        // Nom de l'article
    float prix;          // Prix unitaire en euros
    int quantite;        // Quantite en stock
} Article;
  \end{lstlisting}

  \textbf{Cahier des charges :}
  \begin{itemize}
    \item Demander \textbf{n} ($3 \le n \le 50$) articles
    \item Lire les informations de chaque article
    \item Afficher les 3 articles ayant le \textbf{prix unitaire le plus élevé}
    \item Les afficher par ordre décroissant de prix
    \item Format : \texttt{code - nom: prix EUR}
  \end{itemize}

  \textbf{Indice :} Tri par sélection adapté aux structures.

%   \tcblower
%   \begin{lstlisting}[language=bash]
% check50 IUT-GEII-Annecy/exercices/2025/info1/tp_structures/stock_top3
%   \end{lstlisting}
\end{UPSTIManipulation}

\begin{UPSTIManipulation}{Date — Calcul du nombre de jours écoulés}

  \textbf{Contexte :} Calculer le nombre de jours écoulés depuis le début de l'année pour une date donnée.

  \textbf{Structure à utiliser :}
  \begin{lstlisting}[language=C]
typedef struct {
    int jour;   // 1-31
    int mois;   // 1-12
    int annee;  // ex: 2025
} Date;
  \end{lstlisting}

  \textbf{Cahier des charges :}
  \begin{itemize}
    \item Demander une date (jour, mois, année)
    \item Calculer le nombre de jours écoulés depuis le 1er janvier de la même année
    \item Ne pas tenir compte des années bissextiles (tous les mois ont leur nombre réel de jours)
    \item Afficher le résultat
  \end{itemize}

  \textbf{Exemple :}
  \begin{lstlisting}[style=console]
Jour: 15
Mois: 3
Annee: 2025
Jours ecoules: 74
  \end{lstlisting}

  \textbf{Rappel :} Janvier=31j, Février=28j, Mars=31j, Avril=30j, etc.

%   \tcblower
%   \begin{lstlisting}[language=bash]
% check50 IUT-GEII-Annecy/exercices/2025/info1/tp_structures/date_jours
%   \end{lstlisting}
\end{UPSTIManipulation}

\begin{UPSTIManipulation}{Jeu de cartes — Distribution}

  \textbf{Contexte :} Représenter et manipuler des cartes à jouer.

  \textbf{Structure à utiliser :}
  \begin{lstlisting}[language=C]
typedef struct {
    int valeur;     // 1-13 (As=1, Valet=11, Dame=12, Roi=13)
    char couleur;   // 'C'oeur, 'P'ique, 'T'refle, 'K'arreau
} Carte;
  \end{lstlisting}

  \textbf{Cahier des charges :}
  \begin{itemize}
    \item Créer un tableau de 52 cartes représentant un jeu complet
    \item Afficher toutes les cartes sous la forme \texttt{valeur-couleur}
    \item Par exemple : \texttt{1-C}, \texttt{2-C}, ..., \texttt{13-C}, \texttt{1-P}, etc.
  \end{itemize}

  \textbf{Astuce :} Utiliser deux boucles imbriquées.

%   \tcblower
%   \begin{lstlisting}[language=bash]
% check50 IUT-GEII-Annecy/exercices/2025/info1/tp_structures/cartes_distribution
%   \end{lstlisting}
\end{UPSTIManipulation}

\begin{UPSTIManipulation}{BONUS ULTIME — Mini base de données de contacts}

  \textbf{Contexte :} Créer un système simple de gestion de contacts avec menu.

  \textbf{Structure à utiliser :}
  \begin{lstlisting}[language=C]
typedef struct {
    int id;
    char nom[50];
    char prenom[50];
    char telephone[15];
    char email[50];
} Contact;
  \end{lstlisting}

  \textbf{Cahier des charges :}
  \begin{itemize}
    \item Créer un tableau de maximum 100 contacts
    \item Implémenter un menu avec les options :
    \begin{enumerate}
      \item Ajouter un contact
      \item Afficher tous les contacts
      \item Rechercher un contact par nom
      \item Quitter
    \end{enumerate}
    \item L'ID doit être auto-incrémenté (commence à 1)
    \item La recherche doit afficher tous les contacts dont le nom contient la chaîne recherchée
  \end{itemize}

  \textbf{Défi :} Gérer proprement l'ajout et la recherche !

%   \tcblower
%   \begin{lstlisting}[language=bash]
% check50 IUT-GEII-Annecy/exercices/2025/info1/tp_structures/contacts_bdd
%   \end{lstlisting}
\end{UPSTIManipulation}

\pagebreak
