\section{Projet 14 — Codenames}
\setcounter{UPSTInumeroCahier des charges}{0}

\subsection*{Objectif du jeu}
Mettre en place une version console de \textit{Codenames} permettant de jouer à deux équipes avec un plateau 5×5, un maître‑espion par équipe et une gestion claire des tours, révélations et conditions de fin.

\subsection*{Règles du jeu}
\begin{itemize}
  \item \textbf{Constitution du plateau} : 25 cartes (mots) sont placées en grille 5×5. Sous ces cartes se cachent des rôles : cartes \textbf{équipe Rouge}, cartes \textbf{équipe Bleue}, \textbf{neutres} et une carte \textbf{Assassin}. Une équipe commence (elle a une carte de plus).
  \item \textbf{Rôles} : chaque équipe a un \textbf{maître‑espion} (voit la carte‑clé indiquant la couleur de chaque mot) et des \textbf{agents} (devinent).
  \item \textbf{Tour de jeu} : le maître‑espion donne \textbf{un seul indice} (un mot) suivi d’un \textbf{nombre} indiquant combien de cartes de sa couleur sont visées. Les agents discutent et \textbf{désignent des cartes une par une}. Chaque désignation révèle la carte.
  \item \textbf{Révélations} : si la carte est de la couleur de l’équipe, elle reste au tour et peut continuer jusqu’à nombre + 1 maximum. Si elle est \textbf{neutre}, le tour s’arrête. Si elle est de la \textbf{couleur adverse}, le tour s’arrête et l’autre équipe marque la carte. Si c’est l’\textbf{Assassin}, l’équipe active \textbf{perd immédiatement}.
  \item \textbf{Fin de partie} : dès qu’une équipe a révélé toutes ses cartes, elle \textbf{gagne}. Si l’Assassin est révélé, l’autre équipe \textbf{gagne} aussitôt.
\end{itemize}

\subsection*{Cahier des charges}
\begin{UPSTICahierDesCharges}{1}
\begin{itemize}
  \item Le programme affiche la grille
  \item Le programme demande puis affiche une carte espions pour les que les espions la connaissent
  \item Une touche permet de cacher la carte espion
\end{itemize}
\end{UPSTICahierDesCharges}

\begin{UPSTICahierDesCharges}{2}
\begin{itemize}
  \item On peut choisir une case, la case est alors cachée et remplacée par le numéro de l'équipe à laquelle elle correspond
\end{itemize}
\end{UPSTICahierDesCharges}

\begin{UPSTICahierDesCharges}{3}
\begin{itemize}
  \item Gestion automatique des tours de jeu
\end{itemize}
\end{UPSTICahierDesCharges}
