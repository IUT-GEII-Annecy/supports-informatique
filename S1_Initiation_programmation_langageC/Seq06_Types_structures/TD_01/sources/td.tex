% ==========================================================================

\begin{UPSTIexercice}{Le problème sans structure}
On souhaite gérer les informations d’un étudiant :
\begin{itemize}
  \item son prénom,
  \item son âge,
  \item sa moyenne.
\end{itemize}

\UPSTIquestion{Proposer une liste de variables pour stocker ces informations.}
\UPSTIquestion{Écrire un code qui affiche les informations d'une étudiante nommée « Alice », 20 ans, moyenne 14.2.}
\UPSTIquestion{Comment stocker les informations de 3 étudiants différents ?}
\UPSTIquestion{Pourquoi cette approche devient-elle source d'erreurs ?}
\end{UPSTIexercice}

% ==========================================================================

\begin{UPSTIinfor}{Les types structurés en C}
En C, on peut regrouper plusieurs variables dans une seule entité appelée \textbf{structure}, définie avec le mot-clé \texttt{struct}. 

Dans notre cas, nous allons créer un type structuré qui contiendra différentes informations concernant une même entité. 

La définition d'un type structuré se fait généralement avec le mot-clé \texttt{typedef} pour simplifier son utilisation ultérieure. La syntaxe générale est la suivante :
\begin{lstlisting}[language=C]
typedef struct {
    type1 champ1;
    type2 champ2;
    ...
    typeN champN;
} NomDeLaStructure;
\end{lstlisting}

Prenons l'exemple d'un contact dans un carnet d'adresses. On peut définir une structure \texttt{Contact} qui regroupe plusieurs champs :
\begin{lstlisting}[language=C]
typedef struct {
    char nom[50];
    char prenom[50];
    char telephone[15];
    char email[100];
    int anneeNaissance;
} T_Contact;
\end{lstlisting}
\end{UPSTIinfor}

\begin{UPSTIexercice}{Créer une structure cohérente}
On regroupe les données dans une seule entité : un \textbf{type structuré}.

\UPSTIquestion{A l'aide de l'exemple précédent, écris la définition d'une structure \texttt{Etudiant} contenant les champs correspondants aux informations d'un étudiant (prénom, âge, moyenne).}
\UPSTIquestion{Écrire les déclarations suivantes :}
\begin{itemize}
  \item une variable \texttt{etudiant\_1} représentant un étudiant,
  \item un tableau \texttt{classe} de 30 étudiants.
\end{itemize}
\UPSTIquestion{Ecrire les lignes de code permettant d'initialiser les informations de l'étudiant \texttt{etudiant\_1} avec les valeurs suivantes : prénom « Bob », âge 22, moyenne 15.5.}
\UPSTIquestion{Ecrire les lignes de codes permettant de demander à l'utilisateur de saisir les informations des 30 étudiants du tableau \texttt{classe}.}
\UPSTIquestion{Ecrire les lignes de code permettant d'afficher le nom, prenom et la moyenne du meilleur étudiant de la classe.}
\end{UPSTIexercice}

% ==========================================================================

\begin{UPSTIexercice}{Représenter une heure}
On veut manipuler une heure sous la forme \texttt{HH:MM:SS}.

\UPSTIquestion{Proposer un type structuré \texttt{T\_Heure} permettant de stocker les information d'un horaire.}
\UPSTIquestion{Écrire une fonction \texttt{void afficher\_heure(\texttt{T\_Heure} t)} qui affiche \texttt{HH:MM:SS}.}
\UPSTIquestion{Écrire \texttt{int en\_secondes(T\_Heure)} qui renvoie le nombre total de secondes.}
\UPSTIquestion{Créer deux horaires dans le \texttt{main} et afficher la différence en secondes.}
\end{UPSTIexercice}

% --------------------------------------------------------------------------

\begin{UPSTIexercice}{Gérer un produit en stock}
On veut garder les informations suivantes pour chaque produit :
\begin{itemize}
  \item son nom (chaîne de caractères),
  \item son prix HT (nombre à virgule),
  \item son taux de TVA (nombre à virgule).
\end{itemize}


\UPSTIquestion{Proposer un type structuré \texttt{T\_Produit} pour stocker ces informations.}
\UPSTIquestion{Écrire une fonction qui renvoie le prix TTC d'un produit (la fonction ne prendra qu'un seul argument.)}
\end{UPSTIexercice}

% --------------------------------------------------------------------------

\begin{UPSTIexercice}{Gestion d’un vecteur 2D}
On manipule des vecteurs dans un plan :
\begin{lstlisting}[language=C]
typedef struct{
    float x;
    float y;
} T_Vecteur;
\end{lstlisting}

\UPSTIquestion{Écrire une fonction qui renvoie la norme $\sqrt{x^2 + y^2}$ d'un vecteur donné en paramètre.}
\UPSTIquestion{Écrire une fonction permettant de sommer deux vecteurs.}
\UPSTIquestion{Écrire une fonction peremattant de donner l'opposé d'un vecteur.}
\UPSTIquestion{Dans le \texttt{main}, déclarer trois vecteurs, afficher leurs normes, puis afficher la norme de $a-b$.}
\end{UPSTIexercice}

\begin{UPSTIexercice}{Extension : struct imbriquées}
Il est également possible de créer des structures imbriquées, c'est-à-dire d'utiliser une structure comme champ d'une autre structure. Par exemple, on peut définir une structure \texttt{Adresse} et l'utiliser dans une structure \texttt{Etudiant} :
\begin{lstlisting}[language=C]
typedef struct {
    char rue[50];
    int codePostal;
    char ville[30];
} T_Adresse;

struct Etudiant {
    char prenom[30];
    int age;
    float moyenne;
    T_Adresse adresse;
};
\end{lstlisting}

\UPSTIquestion{Comment accéder à la ville d’un étudiant ?}
\UPSTIquestion{Quel est l’avantage de créer une structure imbriquée plutôt qu’ajouter 3 champs à \texttt{Etudiant} ?}

\end{UPSTIexercice}
