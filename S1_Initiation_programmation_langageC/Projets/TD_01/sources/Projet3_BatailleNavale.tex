\section{Projet 3 — Bataille navale}
\setcounter{UPSTInumeroCahier des charges}{0}

\subsection*{Objectif du jeu}
Couler les bateaux cachés dans une \textbf{mer 6$\times$6} en proposant des \textbf{coordonnées} de tir. Le jeu se joue \textbf{joueur vs processeur}.

\subsection*{Règles du jeu}
\begin{itemize}
  \item La mer est une grille 6$\times$6.
  \item Deux bateaux : \textbf{Bateau n°1} (2 cases) et \textbf{Bateau n°2} (3 cases).
  \item Les bateaux sont placés \textbf{aléatoirement}, \textbf{horizontalement ou verticalement}, sans chevauchement et entièrement dans la grille.
  \item Le joueur propose des coordonnées. \textbf{Touché} si la case contient un bateau, \textbf{Coulé} lorsque toutes les cases d’un même bateau sont touchées.
  \item \textbf{Victoire} quand les deux bateaux sont coulés ; afficher le \textbf{nombre de tentatives} et le \textbf{temps de jeu}.
\end{itemize}

\subsection*{Codage conseillé des cases (pour la logique interne)}
\begin{center}
\begin{tabular}{|c|c|}
\hline
\textbf{Valeur} & \textbf{Signification} \\ \hline
0 & Case vide \\ \hline
1 & Bateau n°1, non touchée \\ \hline
10 & Bateau n°1, touchée \\ \hline
2 & Bateau n°2, non touchée \\ \hline
20 & Bateau n°2, touchée \\ \hline
\end{tabular}
\end{center}

\subsection*{Cahiers des charges}
\begin{UPSTICahierDesCharges}{Grille, placement et tirs}
\begin{itemize}
  \item Le joueur joue contre l'ordinateur et tente de couler ses bateaux.
  \item La mer doit être affichée après chaque tir, avec les cases déjà tirées marquées (Tirées vides / Touchées).
\end{itemize}
\end{UPSTICahierDesCharges}

\begin{UPSTICahierDesCharges}{Deux joueurs}
Un joueur place les bateaux, l'autre tente de les couler. Alterner les rôles.
\end{UPSTICahierDesCharges}

\begin{UPSTICahierDesCharges}{Deux joueurs}
Le vrai jeu. Chaque joueur place ses bateaux en début de partie (sans les montrer à l'autre). Les joueurs jouent à tour de rôle, chacun tentant de couler les bateaux adverses. 
\end{UPSTICahierDesCharges}

\subsection*{Squelette de départ}
\lstinputlisting[language=C]{bataille\_squelette.c}
