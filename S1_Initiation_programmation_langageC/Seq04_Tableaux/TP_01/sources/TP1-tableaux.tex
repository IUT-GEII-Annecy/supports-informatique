\begin{UPSTIManipulation}{Notes — Moyenne de 3 puis 10 notes}
  \textbf{Dossier :} \texttt{tp\_tableaux/01\_moyenne}

  \textbf{Contexte :} Un étudiant souhaite connaître sa moyenne.

  \textbf{Cahier des charges :}
  \begin{itemize}
    \item Demander \textbf{3} notes (float) et afficher la \textbf{moyenne}.
    \item Puis Demander \textbf{10} notes (float) dans un \texttt{float notes[10]} et afficher la \textbf{moyenne}.
    \item Créer une fonction qui calcule la moyenne. Cette fonction prendra deux arguments : un tableau et la taille du tableau.
    \item Modifier le premier programme pour utiliser la fonction.
  \end{itemize}

  \tcblower
  \begin{lstlisting}[language=bash]
check50 IUT-GEII-Annecy/exercices/2025/info1/tp_tableaux/moyenne
  \end{lstlisting}
\end{UPSTIManipulation}


\begin{UPSTIManipulation}{Notes — Filtrer au-dessus d'un seuil}
  \textbf{Dossier :} \texttt{tp\_tableaux/02\_filtre\_notes}

  \textbf{Contexte :} Un professeur veut afficher les notes supérieures à un seuil.

  \textbf{Cahier des charges :}
  \begin{itemize}
    \item Demander le nombre de notes \textbf{n} ($1\le n \le 50$), puis \textbf{n} notes (float) dans \texttt{float notes[50]}.
    \item Demander un \textbf{seuil} (float).
    \item Afficher \textbf{dans l’ordre d’origine} toutes les notes \textbf{strictement supérieures} au seuil,
          ou \texttt{Aucune} si aucune ne correspond.
    \item Afficher également le \textbf{nombre} de notes affichées.
  \end{itemize}

  \tcblower
  \begin{lstlisting}[language=bash]
check50 IUT-GEII-Annecy/exercices/2025/info1/tp_tableaux/filtre_notes
  \end{lstlisting}
\end{UPSTIManipulation}


\begin{UPSTIManipulation}{Cinéma — Places libres et taux d'occupation}
  \textbf{Dossier :} \texttt{tp\_tableaux/03\_cinema\_occupation}

  \textbf{Contexte :} Une rangée de cinéma a \textbf{n} sièges (\texttt{n<=100}). 1 = occupé, 0 = libre.

  \textbf{Cahier des charges :}
  \begin{itemize}
    \item Demander le nombre de siège par rangée \textbf{n} puis \textbf{n} fois si le siège de la rangée est occupé ou non. Utiliser un tableau \texttt{int rang[100]} (0 ou 1).
    \item Afficher les \textbf{indices} des \textbf{places libres} (\texttt{0..n-1}) \textbf{dans l’ordre}.
    \item Calculer et afficher le \textbf{taux d’occupation} en \% (arrondi entier).
    \item Si toutes occupées, afficher \texttt{Complet}.
  \end{itemize}

  \tcblower
  \begin{lstlisting}[language=bash]
check50 IUT-GEII-Annecy/exercices/2025/info1/tp_tableaux/cinema_occupation
  \end{lstlisting}
\end{UPSTIManipulation}


\begin{UPSTIManipulation}{Playlist — Faire commencer à partir d'un morceau favori}
  \textbf{Dossier :} \texttt{tp\_tableaux/04\_playlist\_rotation}

  \textbf{Contexte :} On souhaite \textbf{recycler} une playlist pour qu’elle \textbf{commence} par l’indice \textbf{f} (0..n-1).

  \textbf{Cahier des charges :}
  \begin{itemize}
    \item Lire \textbf{n} ($1\le n \le 100$) puis \textbf{n} noms de fichiers (IDs de morceaux).
    \item Lire \textbf{morceau\_favori} (le nom du morceau favori).
    \item Afficher la playliste en commençant par le morceau favori. Attention, tous les morceaux doivent apparaitre sans modifier l'ordre.
  \end{itemize}

  \textbf{Exemple :}
  \begin{lstlisting}[language=bash]
n=5
Entrer les 5 morceaux :
J adore 
La symphonie des nuages 
La bandite 
Toxicity
Basique

Par quel morceau commencer : La bandite
Merci. Nouvel ordre :
La bandite ; Toxicity ; Basique ; J adore ; La symphonie des nuages
  \end{lstlisting}

  \tcblower
  \begin{lstlisting}[language=bash]
check50 IUT-GEII-Annecy/exercices/2025/info1/tp_tableaux/playlist_rotation
  \end{lstlisting}
\end{UPSTIManipulation}


\begin{UPSTIManipulation}{Playlist — Retirer les doublons (ordre conservé)}
  \textbf{Dossier :} \texttt{tp\_tableaux/05\_playlist\_dedupe}

  \textbf{Contexte :} Une playlist contient des IDs \textbf{répétés}. On veut garder \textbf{seulement la première occurrence} de chaque ID.

  \textbf{Cahier des charges :}
  \begin{itemize}
    \item Lire \textbf{n} ($1\le n \le 100$) puis \textbf{n} entiers \texttt{ids}.
    \item Construire \texttt{out} \textbf{sans doublons} en conservant l’ordre d’apparition.
    \item Afficher la \textbf{taille} finale puis les éléments.
    \item Interdit : trier.
  \end{itemize}

  \textbf{Exemple :}
  \begin{lstlisting}[language=bash]
n=7
ids= 4 9 4 2 9 1 2
taille=4
-> 4 9 2 1
  \end{lstlisting}

  \tcblower
  \begin{lstlisting}[language=bash]
check50 IUT-GEII-Annecy/exercices/2025/info1/tp_tableaux/playlist_dedupe
  \end{lstlisting}
\end{UPSTIManipulation}


\begin{UPSTIManipulation}{Borner le nombre d'invités}
  \textbf{Dossier :} \texttt{tp\_tableaux/06\_stock\_clamp}

  \textbf{Contexte :} On demande à plusieurs personnes combien d'invités elles désirent amener à une fête. Cependant, on demande aux invités d'amener au moins \textbf{minQ} et au plus \textbf{maxQ} invités. On veut donc corriger les quantités demandées.

  \textbf{Cahier des charges :}
  \begin{itemize}
    \item Demander le \textbf{nombre\_participants}  puis, pour chacun, le nombre d'invités souhaités. (Stocker ces nombres dans un tableau \texttt{qte[100]}).
    \item Demander \textbf{minQ} et \textbf{maxQ} (Vérifier que \texttt{minQ <= maxQ}).
    \item \textbf{Modifier en place} : toute valeur $<minQ$ devient \texttt{minQ}, et $>maxQ$ devient \texttt{maxQ}.
    \item Afficher le tableau corrigé.
  \end{itemize}

  \tcblower
  \begin{lstlisting}[language=bash]
check50 IUT-GEII-Annecy/exercices/2025/info1/tp_tableaux/stock_clamp
  \end{lstlisting}
\end{UPSTIManipulation}


\begin{UPSTIManipulation}{Web analytics — Moyenne glissante (fenêtre 3)}
  \textbf{Dossier :} \texttt{tp\_tableaux/07\_web\_moving\_avg}

  \textbf{Contexte :} On lisse une série de \textbf{visites par heure} avec une moyenne glissante.

  \textbf{Cahier des charges :}
  \begin{itemize}
    \item Lire \textbf{n} puis \textbf{n} entiers dans \texttt{visites[100]}.
    \item Produire \texttt{out[i]} :
      \begin{itemize}
        \item \texttt{out[0] = visites[0]}, \texttt{out[n-1] = visites[n-1]}
        \item pour $1 \le i \le n-2$ : \texttt{out[i] = (v[i-1]+v[i]+v[i+1]) / 3} (division entière)
      \end{itemize}
    \item Afficher \texttt{out}.
  \end{itemize}

  \textbf{Exemple :}
  \begin{lstlisting}[language=bash]
n=5
visites= 2 10 5 7 1
out= 2 5 7 4 1
  \end{lstlisting}

  \textbf{Types :} \texttt{int}. Deux tableaux \texttt{v[100]}, \texttt{out[100]}.

  \tcblower
  \begin{lstlisting}[language=bash]
check50 IUT-GEII-Annecy/exercices/2025/info1/tp_tableaux/web_moving_avg
  \end{lstlisting}
\end{UPSTIManipulation}


\begin{UPSTIManipulation}{Tarifs — Tri par sélection croissant}
  \textbf{Dossier :} \texttt{tp\_tableaux/08\_tarifs\_selection\_sort}

  \textbf{Contexte :} Un cinéma veut trier une liste de \textbf{tarifs en centimes} pour affichage.

  \textbf{Cahier des charges :}
  \begin{itemize}
    \item Lire \textbf{n} puis \textbf{n} entiers (centimes) \texttt{p[100]}.
    \item Trier \textbf{par sélection} croissant (implémentation perso).
    \item Afficher le tableau trié et le \textbf{nombre d’échanges} effectués.
  \end{itemize}

  \textbf{Exemple :}
  \begin{lstlisting}[language=bash]
n=6
p= 500 200 400 600 100 300
sorted= 100 200 300 400 500 600
swaps= 4
  \end{lstlisting}

  \textbf{Types :} \texttt{int}. Pas de \texttt{qsort}.

  \tcblower
  \begin{lstlisting}[language=bash]
check50 IUT-GEII-Annecy/exercices/2025/info1/tp_tableaux/tarifs_selection_sort
  \end{lstlisting}
\end{UPSTIManipulation}


\begin{UPSTIManipulation}{Catalogue — Recherche dichotomique par ID}
  \textbf{Dossier :} \texttt{tp\_tableaux/09\_catalogue\_bsearch}

  \textbf{Contexte :} Les IDs d’un catalogue sont \textbf{triés} croissants. On veut retrouver l’indice d’un ID donné.

  \textbf{Cahier des charges :}
  \begin{itemize}
    \item Lire \textbf{n} puis \textbf{n} entiers triés \texttt{id[100]}.
    \item Lire \textbf{x} (ID recherché).
    \item \textbf{Recherche binaire} (boucle \texttt{while}) : afficher l’indice ou \texttt{-1}.
  \end{itemize}

  \textbf{Exemples :}
  \begin{lstlisting}[language=bash]
id= 10 20 30 40 50, x=40 -> 3
id= 2 4 6 8, x=5 -> -1
  \end{lstlisting}

  \textbf{Types :} \texttt{int}. Contrôler \texttt{low}, \texttt{high}, \texttt{mid}.

  \tcblower
  \begin{lstlisting}[language=bash]
check50 IUT-GEII-Annecy/exercices/2025/info1/tp_tableaux/catalogue_bsearch
  \end{lstlisting}
\end{UPSTIManipulation}


\begin{UPSTIManipulation}{Qualité — Seuillage d'image (matrice 2D)}
  \textbf{Dossier :} \texttt{tp\_tableaux/10\_image\_seuillage\_2D}

  \textbf{Contexte :} Une caméra industrielle renvoie une image en niveaux de gris (0..255). On veut un \textbf{masque binaire}.

  \textbf{Cahier des charges :}
  \begin{itemize}
    \item Lire \textbf{h}, \textbf{w} ($1\le h,w \le 20$), puis \textbf{h*w} entiers dans \texttt{M[20][20]}.
    \item Lire un \textbf{seuil T}.
    \item Produire \texttt{B[i][j] = 1} si \texttt{M[i][j] >= T}, sinon \texttt{0}.
    \item Afficher \texttt{B} en grille (0/1) avec un espace entre colonnes.
  \end{itemize}

  \textbf{Exemple :}
  \begin{lstlisting}[language=bash]
h=2, w=3, T=100
M =
  0 120 255
  80 100  90
B =
  0 1 1
  0 1 0
  \end{lstlisting}

  \textbf{Types :} \texttt{int}. Deux boucles imbriquées.

  \tcblower
  \begin{lstlisting}[language=bash]
check50 IUT-GEII-Annecy/exercices/2025/info1/tp_tableaux/image_seuillage_2D
  \end{lstlisting}
\end{UPSTIManipulation}
