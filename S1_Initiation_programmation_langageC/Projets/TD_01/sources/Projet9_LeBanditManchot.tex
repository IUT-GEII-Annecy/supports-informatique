\section{Projet 9 — Le bandit manchot}
\setcounter{UPSTInumeroCahier des charges}{0}

\subsection*{Objectif du jeu}
Simuler un bandit manchot à 3 rouleaux et gérer le capital du joueur selon les combinaisons obtenues.

\subsection*{Règles du jeu}
\begin{itemize}
  \item 3 \textbf{rouleaux} de 11 figures (tableau 2D). Codes : \texttt{7, B, L, C} (sept, barre, citron, cerise).
  \item Le joueur commence avec \textbf{100 points} et mise 1 ou 3 points.
  \item Appui sur \texttt{J} : tirage aléatoire d’une figure par rouleau; affichage en ligne des trois codes.
  \item Gagne/perd selon les combinaisons définies; le capital est mis à jour; le joueur peut continuer (\texttt{O/N}).
\end{itemize}

\subsection*{Cahier des charges}
\begin{UPSTICahierDesCharges}{1}
\begin{itemize}
  \item Initialiser les \textbf{3 rouleaux} (tableau 3×11) et afficher leur contenu pour vérification.
  \item Tirer et afficher \textbf{une figure par rouleau} (codes).
  \item Mettre à jour le capital selon résultat simple (ex.\,égalité des 3).
\end{itemize}
\end{UPSTICahierDesCharges}

\begin{UPSTICahierDesCharges}{2}
\begin{itemize}
  \item Implémenter les \textbf{mises} (1 ou 3), détailler les combinaisons gagnantes/perdantes.
  \item Autoriser de \textbf{conserver} un ou plusieurs rouleaux (ex.\,saisie \texttt{23}) avec pénalité de points.
  \item Récapitulatif clair de fin.
\end{itemize}
\end{UPSTICahierDesCharges}

\begin{UPSTICahierDesCharges}{3}
\begin{itemize}
  \item Présentation et rejouabilité; options de test (rouleaux biaisés).
  \item Structuration en fonctions (tirage, calcul gain, affichage).
  \item Robustesse des saisies.
\end{itemize}
\end{UPSTICahierDesCharges}

\subsection*{Squelette de départ}
\lstinputlisting[language=C]{bandit_squelette.c}
