\section{Projet 5 — Jeu des couleurs}
\setcounter{UPSTInumeroCahier des charges}{0}

\subsection*{Objectif du jeu}
L’ordinateur tire une \textbf{suite de 10 couleurs} parmi \textit{noir, vert, rouge, bleu}. Les couleurs s’affichent brièvement, puis disparaissent. Le joueur doit \textbf{reproduire la séquence} dans le bon ordre (codage numérique). Le score et la durée sont affichés en fin de partie. Un mode 2 joueurs est ensuite proposé.

\subsection*{Règles du jeu}
\begin{itemize}
  \item L'ordinateur choisit 10 couleurs au hasard parmi \{noir, vert, rouge, bleu\}. 
  \item Les 10 carrés de couleurs s'affichent en ligne pendant 5\,s, puis l’écran est effacé.
  \item Le joueur saisit la séquence en codes : \texttt{0} pour noir, \texttt{1} pour bleu, \texttt{2} pour vert, \texttt{3} pour rouge.
  \item À chaque saisie correcte, \textbf{+1 point}. À chaque erreur, \textbf{-1 point}.
  \item En fin de partie : Afficher le \textbf{score}.
\end{itemize}

\subsection*{Cahier des charges}
\begin{UPSTICahierDesCharges}{Mode 1 joueur simplifié}
\begin{itemize}
  \item Affichage de la séquence, puis effacement de l'écran
  \item Le joueur tente de reformer la suite
  \item Affichage du score en fin de partie.
\end{itemize}
\textit{Aide :} utiliser \texttt{srand(time(NULL))}, \texttt{rand()\%4}, \texttt{sleep(5)}, \texttt{system("clear")}; ANSI \texttt{\textbackslash033[40m..\textbackslash033[0m} pour les carrés.
\end{UPSTICahierDesCharges}

\begin{UPSTICahierDesCharges}{Mode 1 joueur complet}
\begin{itemize}
  \item A la fin de la partie, la séquence originale et la séquence du joueur sont affichées pour comparaison
  \item La durée et le score sont également affichés
\end{itemize}
\end{UPSTICahierDesCharges}

\begin{UPSTICahierDesCharges}{Mode 2 joueurs et bilan complet}
\begin{itemize}
  \item Chaque joueur saisit sa propre \textbf{séquence} ; comparer les scores et \textbf{annoncer le vainqueur}.
  \item Afficher les \textbf{trois séquences} : tirage, joueur~1, joueur~2, ainsi que leurs scores et durées.
  \item Soigner la \textbf{présentation} (espaces, retours à la ligne, lisibilité des carrés).
\end{itemize}
\textit{Aide :} réutiliser l’affichage des carrés et une fonction de calcul de score pour éviter la duplication.
\end{UPSTICahierDesCharges}

\subsection*{Squelette de départ}
\lstinputlisting[language=C]{couleurs\_squelette.c}
