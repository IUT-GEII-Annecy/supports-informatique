
\begin{UPSTIinfor}{Les types structurés en C}
  Un \textbf{type structuré} (ou \texttt{struct}) permet de regrouper plusieurs données de types différents en une seule entité logique.

  \textbf{Avantages :}
  \begin{itemize}
    \item Regrouper des données liées ensemble (cohésion)
    \item Créer des types de données personnalisés
    \item Améliorer la lisibilité du code
    \item Faciliter la manipulation de données complexes
  \end{itemize}

  \textbf{Syntaxe de déclaration :}
  \begin{lstlisting}[language=C]
struct NomStructure {
    type1 champ1;
    type2 champ2;
    // ...
};
  \end{lstlisting}

  \textbf{Exemple concret :}
  \begin{lstlisting}[language=C]
struct Point {
    float x;
    float y;
};
  \end{lstlisting}
\end{UPSTIinfor}

\begin{UPSTIinfor}{Utilisation d'une structure}
  \textbf{Création d'une variable de type structure :}
  \begin{lstlisting}[language=C]
struct Point p1;  // Déclare une variable p1 de type struct Point
  \end{lstlisting}

  \textbf{Accès aux champs :}
  \begin{lstlisting}[language=C]
p1.x = 3.5;       // Affecte 3.5 au champ x
p1.y = 2.7;       // Affecte 2.7 au champ y
printf("Point: (%.1f, %.1f)\n", p1.x, p1.y);
  \end{lstlisting}

  \textbf{Initialisation lors de la déclaration :}
  \begin{lstlisting}[language=C]
struct Point p2 = {1.0, 2.0};  // x=1.0, y=2.0
  \end{lstlisting}
\end{UPSTIinfor}

\begin{UPSTIinfor}{Typedef : simplifier l'utilisation}
  Le mot-clé \texttt{typedef} permet de créer un alias pour une structure, simplifiant son utilisation.

  \textbf{Sans typedef :}
  \begin{lstlisting}[language=C]
struct Point {
    float x;
    float y;
};
struct Point p1;  // On doit écrire "struct Point"
  \end{lstlisting}

  \textbf{Avec typedef :}
  \begin{lstlisting}[language=C]
typedef struct {
    float x;
    float y;
} Point;

Point p1;  // Plus simple ! Pas besoin de "struct"
  \end{lstlisting}
\end{UPSTIinfor}

\begin{UPSTIinfor}{Structures et fonctions}
  Les structures peuvent être passées en paramètres et retournées par des fonctions.

  \textbf{Passage par valeur (copie) :}
  \begin{lstlisting}[language=C]
void afficher_point(Point p) {
    printf("(%.1f, %.1f)\n", p.x, p.y);
}
  \end{lstlisting}

  \textbf{Passage par adresse (modification possible) :}
  \begin{lstlisting}[language=C]
void deplacer_point(Point *p, float dx, float dy) {
    p->x += dx;  // Notation -> pour accéder aux champs via pointeur
    p->y += dy;
}
  \end{lstlisting}

  \textbf{Retour de structure :}
  \begin{lstlisting}[language=C]
Point creer_point(float x, float y) {
    Point p;
    p.x = x;
    p.y = y;
    return p;
}
  \end{lstlisting}
\end{UPSTIinfor}

\pagebreak
