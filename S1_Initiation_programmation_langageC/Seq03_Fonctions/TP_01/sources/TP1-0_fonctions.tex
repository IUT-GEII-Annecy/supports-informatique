
\section{Premières fonctions en C}

\begin{UPSTIManipulation}{Une fonction qui affiche une ligne}
	Reprendre le programme carre qui affiche un carré d'étoiles.
	Créer une fonction \texttt{void afficher\_ligne(int n)} qui affiche une ligne de \texttt{n} étoiles.
	Utiliser cette fonction dans le programme principal pour afficher le carré.
\end{UPSTIManipulation}

\begin{UPSTIManipulation}{Afficher des espaces}
	Ecrire une fonction \texttt{void afficher\_espaces(int n)} qui affiche \texttt{n} espaces.

	Utiliser cette fonction et la précédente pour afficher un triangle d'étoiles isocèle et centré, de hauteur \texttt{h} (entier entré par l'utilisateur).
\end{UPSTIManipulation}

\begin{UPSTIManipulation}{Débuggeur}
	Utiliser le débuggeur (debug50) pour observer l'évolution du programme. 
\end{UPSTIManipulation}

\begin{UPSTIManipulation}{Une fonction qui demande un entier positif}
	Ecrire une fonction \texttt{int demander\_entier\_positif(void)} qui demande à l'utilisateur de saisir un entier positif, et qui répète la demande tant que l'utilisateur n'entre pas un entier positif.
	Utiliser cette fonction dans un programme principal pour récupérer un entier positif et l'afficher.
\end{UPSTIManipulation}

\begin{UPSTIManipulation}{Fonctions visuelles}Reprendre tous les exercices, à partir de la Manipulation 10 du TP précédent, et utiliser  des fonctions pour les rendre plus lisibles
\end{UPSTIManipulation}

\begin{UPSTIManipulation}{Calculer le maximum}
	Ecrire une fonction \texttt{maximum} qui prend en entrée deux paramètres : \texttt{a} et \texttt{b} et qui retourne la valeur maximale entre ces deux nombres.

	Méthode : 
	\begin{enumerate}
		\item Dessiner la boite représentant la fonction
		\item Ecrire le prototype de la fonction
		\item Ecrire le corps de la fonction.
	\end{enumerate}
\end{UPSTIManipulation}

\begin{UPSTIManipulation}{Températures}
	Ecrire deux fonctions \texttt{celcius\_vers\_fahrenheit} et \texttt{fahrenheit\_vers\_celcius} qui réalisent les conversions de températures.
\end{UPSTIManipulation} 



 
