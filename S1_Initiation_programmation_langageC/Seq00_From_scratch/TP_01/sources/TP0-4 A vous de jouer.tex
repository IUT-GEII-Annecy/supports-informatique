\section{A vous de jouer !}
A vous de créer un programme Scratch qui vous ressemble et qui montre vos compétences en programmation.

Ce premier rendu peut être un jeu, une animation, une histoire interactive, ou tout autre projet qui vous inspire. Les seules contraintes sont : 
\begin{UPSTIaRendre}{Projet Scratch}    
    \begin{itemize}
    \item Votre projet doit utiliser au moins deux lutins (sprites), dont au moins un ne doit pas être un chat.
    \item Votre projet doit comporter au moins trois scripts au total (c'est-à-dire pas nécessairement trois par sprites).
    \item Votre projet doit utiliser au moins une conditionnelle, au moins une boucle et au moins une variable.
    \item Votre projet doit utiliser au moins un bloc personnalisé que vous avez créé vous-même (via Créer un bloc), qui doit prendre au moins une entrée.
    \item Votre projet doit être interactif, c'est-à-dire que l'utilisateur doit pouvoir interagir avec le programme d'une manière ou d'une autre (par exemple, en cliquant sur un sprite, en appuyant sur une touche, etc.).
    \end{itemize}
\end{UPSTIaRendre}

Le rendu se fera sur la plateforme moodle, où vous devrez soumettre le lien vers votre projet Scratch.  

 


\begin{UPSTIinfor}{En manque d'idées ?}
    Voici quelques propositions de projets, classées par niveau de difficulté. 
    
    \begin{description}
        \item[Facile :]  Pour les moins à l'aise 
        \begin{itemize}
            \item Une balle qui rebondit sur les bords de l’écran.  
            \item Deux personnages qui racontent une histoire
            \item Un mini-piano où chaque touche du clavier joue une note.  
        \end{itemize}

        \item[Intermédiaire :] Si vous comprenez bien comment fonctionne Scratch
        \begin{itemize}
            \item Une course entre deux lutins contrôlés par le clavier.  
            \item Un petit quizz à choix multiples.  
            \item Une histoire interactive où l’utilisateur choisit la suite du récit.  
        \end{itemize}

        \item[Avancé :]  Pour les plus motivés !
        \begin{itemize}
            \item Un jeu de labyrinthe avec un chronomètre.  
            \item Un simulateur de feu d’artifice avec sons et effets visuels.  
            \item Un jeu de plateforme avec plusieurs niveaux.  
            \item Un jeu de combat entre deux personnages avec points de vie.  
        \end{itemize}
    \end{description}
\end{UPSTIinfor}