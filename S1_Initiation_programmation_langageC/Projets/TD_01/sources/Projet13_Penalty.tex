\section{Projet 13 — Penalty !}
\setcounter{UPSTInumeroCahier des charges}{0}

\subsection*{Objectif du jeu}
Simuler une séance de tirs au but avec choix du tireur et du gardien, comptage des buts et affichage du score.

\subsection*{Règles du jeu}
\begin{itemize}
  \item Deux adversaires s’affrontent au \textbf{tir au but}. À chaque tir, le tireur choisit une zone, le gardien choisit une zone.
  \item But si les choix ne coïncident pas; sinon arrêt. Après une série, on affiche le \textbf{score} et le vainqueur.
  \item Variantes possibles (meilleure des 5, mort subite, difficulté).
\end{itemize}

\subsection*{Cahier des charges}
\begin{UPSTICahierDesCharges}{1}
\begin{itemize}
  \item Version \textbf{basique jouable} : saisies tireur/gardien, détection but/arrêt, \textbf{compteur de buts} et fin de série.
  \item Affichage du \textbf{score} et du vainqueur.
  \item Rejouer une série.
\end{itemize}
\end{UPSTICahierDesCharges}

\begin{UPSTICahierDesCharges}{2}
\begin{itemize}
  \item Validation des saisies, modes de série (5 tirs, mort subite).
  \item Ajout d’une \textbf{IA} simple pour le gardien.
  \item Récapitulatif de fin.
\end{itemize}
\end{UPSTICahierDesCharges}

\begin{UPSTICahierDesCharges}{3}
\begin{itemize}
  \item Présentation et rejouabilité, statistiques.
  \item Facteur aléatoire (puissance tir, précision) optionnel.
  \item Factoriser en fonctions.
\end{itemize}
\end{UPSTICahierDesCharges}

