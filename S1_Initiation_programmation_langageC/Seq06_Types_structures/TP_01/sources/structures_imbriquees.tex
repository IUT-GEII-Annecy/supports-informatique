Cette partie introduit les structures imbriquées et des applications plus réalistes.

\subsection{Structures contenant d'autres structures}

\begin{UPSTIManipulation}{Rectangle — Calcul de l'aire}

  \textbf{Contexte :} Un rectangle est défini par deux points : coin supérieur gauche et coin inférieur droit.

  \textbf{Structures à utiliser :}
  \begin{lstlisting}[language=C]
typedef struct {
    float x;
    float y;
} Point;

typedef struct {
    Point coin_haut_gauche;
    Point coin_bas_droit;
} Rectangle;
  \end{lstlisting}

  \textbf{Cahier des charges :}
  \begin{itemize}
    \item Demander les coordonnées des deux coins du rectangle
    \item Calculer l'aire : $\text{largeur} \times \text{hauteur}$
    \item Largeur = $|x_2 - x_1|$, Hauteur = $|y_2 - y_1|$
    \item Afficher l'aire avec 2 décimales
  \end{itemize}

  \textbf{Exemple :}
  \begin{lstlisting}[style=console]
Coin haut-gauche - x: 0
Coin haut-gauche - y: 3
Coin bas-droit - x: 4
Coin bas-droit - y: 0
Aire: 12.00
  \end{lstlisting}

%   \tcblower
%   \begin{lstlisting}[language=bash]
% check50 IUT-GEII-Annecy/exercices/2025/info1/tp_structures/rectangle_aire
%   \end{lstlisting}
\end{UPSTIManipulation}

\begin{UPSTIManipulation}{Cercle — Intersection avec un point}

  \textbf{Contexte :} Vérifier si un point est à l'intérieur d'un cercle.


  \textbf{Cahier des charges :}
  \begin{itemize}
    \item Proposer une structure pour représenter un cercle.
    \item Demander le centre du cercle (x, y) et son rayon
    \item Demander les coordonnées d'un point à tester
    \item Calculer la distance entre le point et le centre
    \item Afficher \texttt{INTERIEUR} si distance $\le$ rayon, sinon \texttt{EXTERIEUR}
  \end{itemize}

  \textbf{Exemple :}
  \begin{lstlisting}[style=console]
Centre du cercle - x: 0
Centre du cercle - y: 0
Rayon: 5
Point a tester - x: 3
Point a tester - y: 4
INTERIEUR
  \end{lstlisting}

  \textbf{Note :} Distance = 5, exactement sur le cercle, donc INTERIEUR.

%   \tcblower
%   \begin{lstlisting}[language=bash]
% check50 IUT-GEII-Annecy/exercices/2025/info1/tp_structures/cercle_point
%   \end{lstlisting}
\end{UPSTIManipulation}

% \subsection{Application réaliste}

% \begin{UPSTIManipulation}{Gestion d'étudiants — Calcul de moyenne}

%   \textbf{Contexte :} Gérer une liste d'étudiants avec leurs notes.

%   \textbf{Structure à utiliser :}
%   \begin{lstlisting}[language=C]
% typedef struct {
%     int numero;          // Numero etudiant
%     char nom[50];        // Nom de l'etudiant
%     float note1;         // Premiere note
%     float note2;         // Deuxieme note
%     float note3;         // Troisieme note
% } Etudiant;
%   \end{lstlisting}

%   \textbf{Cahier des charges :}
%   \begin{itemize}
%     \item Demander \textbf{n} ($1 \le n \le 30$) étudiants
%     \item Pour chaque étudiant, lire : numéro, nom, 3 notes
%     \item Pour chaque étudiant, calculer et afficher sa moyenne (2 décimales)
%     \item Format : \texttt{Etudiant numero - nom: moyenne}
%   \end{itemize}

%   \textbf{Exemple :}
%   \begin{lstlisting}[style=console]
% Nombre d'etudiants: 2
% Etudiant 0 - Numero: 12345
% Nom: Dupont
% Note 1: 15.5
% Note 2: 14.0
% Note 3: 16.5
% Etudiant 1 - Numero: 12346
% Nom: Martin
% Note 1: 12.0
% Note 2: 13.5
% Note 3: 11.5

% Moyennes:
% Etudiant 12345 - Dupont: 15.33
% Etudiant 12346 - Martin: 12.33
%   \end{lstlisting}

%   \tcblower
%   \begin{lstlisting}[language=bash]
% check50 IUT-GEII-Annecy/exercices/2025/info1/tp_structures/etudiants_moyenne
%   \end{lstlisting}
% \end{UPSTIManipulation}

% \begin{UPSTIManipulation}{Étudiants — Classement par moyenne}

%   \textbf{Contexte :} Afficher les étudiants dont la moyenne dépasse un seuil donné.

%   \textbf{Cahier des charges :}
%   \begin{itemize}
%     \item Utiliser la même structure \texttt{Etudiant}
%     \item Demander \textbf{n} étudiants et les lire
%     \item Demander un \textbf{seuil} de moyenne
%     \item Afficher les étudiants dont la moyenne est \textbf{supérieure ou égale} au seuil
%     \item Format : \texttt{numero - nom: moyenne}
%     \item Si aucun étudiant ne dépasse le seuil, afficher \texttt{Aucun etudiant}
%   \end{itemize}

%   \tcblower
%   \begin{lstlisting}[language=bash]
% check50 IUT-GEII-Annecy/exercices/2025/info1/tp_structures/etudiants_classement
%   \end{lstlisting}
% \end{UPSTIManipulation}

