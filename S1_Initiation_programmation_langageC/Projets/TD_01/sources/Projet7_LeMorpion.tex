\section{Projet 7 — Le morpion}
\setcounter{UPSTInumeroCahier des charges}{0}

\subsection*{Objectif du jeu}
Réaliser un morpion (3×3) jouable à deux joueurs au clavier, avec affichage ASCII et détection de victoire/égalité.

\subsection*{Règles du jeu}
\begin{itemize}
  \item La grille est un tableau de \textbf{9 cases} : \texttt{0} = vide, \texttt{1} = 'X' (J1), \texttt{2} = 'O' (J2).
  \item Les joueurs jouent à tour de rôle. \textbf{Même code} pour les deux joueurs (variable joueur).
  \item Un joueur gagne s’il aligne 3 symboles en ligne, colonne ou diagonale. Sinon, partie nulle.
\end{itemize}

\subsection*{Cahier des charges}
\begin{UPSTICahierDesCharges}{1}
\begin{itemize}
  \item Boucle principale avec alternance J1/J2, \textbf{saisie} d’une case et \textbf{affichage} de la grille.
  \item Affecter 1 ou 2 selon le joueur, sans validation initiale.
  \item Afficher la grille après chaque coup.
\end{itemize}
\end{UPSTICahierDesCharges}

\begin{UPSTICahierDesCharges}{2}
\begin{itemize}
  \item \textbf{Valider} la saisie (case existante et libre), sinon rejouer.
  \item \textbf{Détecter} la victoire (lignes, colonnes, diagonales) et l’égalité.
  \item Présenter un \textbf{récapitulatif} en fin de partie.
\end{itemize}
\end{UPSTICahierDesCharges}

\begin{UPSTICahierDesCharges}{3}
\begin{itemize}
  \item Ajouter un \textbf{mode IA} basique (ne pas perdre) ou des options (rejouer).
  \item Structurer le code (fonctions d’affichage et de test).
\end{itemize}
\end{UPSTICahierDesCharges}

\subsection*{Squelette de départ}
\lstinputlisting[language=C]{morpion_squelette.c}
