\subsection*{Objectifs du TP}

Ce TP vous permet de découvrir et manipuler les \textbf{types structurés} (structures) en langage C. Les structures sont un concept fondamental qui permet de regrouper plusieurs données de types différents en une seule entité logique.

\textbf{Compétences visées :}
\begin{itemize}
  \item Déclarer et utiliser des structures simples
  \item Manipuler des tableaux de structures
  \item Créer des structures imbriquées
  \item Passer des structures en paramètres de fonctions
  \item Résoudre des problèmes réalistes avec les structures
\end{itemize}

% \subsection*{Organisation du TP}

% Le TP est organisé en 4 parties progressives :
% \begin{enumerate}
%   \item \textbf{Premiers pas} : Découverte des structures avec des points 2D
%   \item \textbf{Tableaux de structures} : Manipulation de collections de données structurées
%   \item \textbf{Structures imbriquées} : Structures contenant d'autres structures
%   \item \textbf{Bonus} : Exercices plus avancés pour aller plus loin
% \end{enumerate}

\subsection*{Validation des exercices}

Les premiers exercices peuvent être validés avec l'outil \texttt{check50} :
\begin{lstlisting}[language=bash]
check50 IUT-GEII-Annecy/exercices/2025/info1/tp_structures/nom_exercice
\end{lstlisting}

\textbf{Conseils :}
\begin{itemize}
  \item Testez votre code manuellement avant d'utiliser \texttt{check50}
  \item Lisez attentivement les messages d'erreur
  \item N'hésitez pas à utiliser le débogueur (\texttt{gdb} ou l'extension VS Code)
  \item Demandez de l'aide à votre enseignant si vous êtes bloqué
\end{itemize}

\subsection*{Conseils méthodologiques}

Pour réussir ce TP :
\begin{enumerate}
  \item \textbf{Lisez la théorie} avant de commencer les exercices
  \item \textbf{Commencez simple} : faites les premiers exercices même s'ils semblent faciles
  \item \textbf{Testez progressivement} : compilez et testez après chaque modification
  \item \textbf{Utilisez des noms explicites} : nommez vos variables et fonctions clairement
  \item \textbf{Commentez votre code} : cela vous aidera à comprendre votre propre logique
\end{enumerate}

\textbf{Durée estimée :} \UPSTIduree

\vspace{1cm}

\begin{center}
  \textbf{Bon travail !}
\end{center}

\pagebreak
