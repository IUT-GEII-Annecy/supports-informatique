\begin{UPSTIexercice}{Identifier les tableaux}
    Voici un extrait de programme :
    \begin{lstlisting}[language=C]
#include <stdio.h>

int main(void) {
    int notes[5] = {12, 15, 9, 14, 18};
    printf("%d\n", notes[2]);
}
\end{lstlisting}

    \UPSTIquestion{Combien d'éléments contient le tableau \texttt{notes} ?}
    \UPSTIquestion{Quel est le type de ses éléments ?}
    \UPSTIquestion{Que va afficher le programme ?}
    \UPSTIquestion{Quel est l’indice du premier élément ? Et du dernier ?}
\end{UPSTIexercice}


\begin{UPSTIexercice}{Remplir un tableau}
    On veut lire 5 entiers au clavier et les stocker dans un tableau.

    \UPSTIquestion{Déclarer un tableau capable de contenir 5 entiers.}
    \UPSTIquestion{Écrire une boucle \texttt{for} permettant de saisir les valeurs au clavier.}
    \UPSTIquestion{Afficher ensuite le contenu du tableau, séparé par des espaces.}
    \UPSTIquestion{Tester le programme avec des valeurs connues.}
\end{UPSTIexercice}


\begin{UPSTIexercice}{Somme et moyenne}
    On dispose d’un tableau \texttt{notes[5]} contenant des entiers.

    \UPSTIquestion{Écrire un programme qui calcule la somme de ses éléments.}
    \UPSTIquestion{En déduire la moyenne.}
\end{UPSTIexercice}


\begin{UPSTIexercice}{Trouver le maximum}
    \UPSTIquestion{Écrire un programme qui parcourt un tableau d’entiers et affiche la plus grande valeur contenue.}
    \UPSTIquestion{Que se passe-t-il si tous les éléments sont égaux ?}
    \UPSTIquestion{Proposer une variable \texttt{indiceMax} pour mémoriser l’indice du maximum et l’afficher à la fin.}
\end{UPSTIexercice}


\begin{UPSTIexercice}{Inverser un tableau}
    \UPSTIquestion{Écrire un programme qui recopie un tableau dans un autre tableau, mais dans l’ordre inverse.}
    \UPSTIquestion{Afficher le tableau d’origine et le tableau inversé.}
\end{UPSTIexercice}


\begin{UPSTIexercice}{Recherche dans un tableau}
    On souhaite savoir si une valeur donnée se trouve dans un tableau.

    \UPSTIquestion{Écrire une fonction \texttt{int contient(int t[], int n, int valeur)} qui renvoie 1 si la valeur est présente, 0 sinon.}
    \UPSTIquestion{Modifier la fonction pour qu’elle renvoie l’indice de la valeur recherchée, ou -1 si elle n’existe pas.}
\end{UPSTIexercice}


\begin{UPSTIexercice}{Moyenne d'une classe}
    On dispose d’un tableau contenant les notes d’une classe de 30 étudiants.

    \UPSTIquestion{Écrire une fonction \texttt{float moyenne(int notes[], int n)}.}
    \UPSTIquestion{Écrire une autre fonction \texttt{int au\_dessus\_de\_moyenne(int notes[], int n)} qui compte le nombre d’étudiants au-dessus de la moyenne.}
    \UPSTIquestion{Comment tester rapidement ces fonctions dans le \texttt{main} ?}
\end{UPSTIexercice}


\begin{UPSTIexercice}{Tableau de caractères}
    \UPSTIquestion{Créer une chaîne de caractères \texttt{char mot[] = "bonjour";}}
    \UPSTIquestion{Quelle est la longueur de ce tableau ? Expliquer.}
    \UPSTIquestion{Afficher chaque caractère un par un avec une boucle.}
    \UPSTIquestion{Modifier le programme pour inverser la chaîne.}
\end{UPSTIexercice}


\begin{UPSTIexercice}{Défi — Histogramme de notes}
    \begin{minipage}{.8\linewidth}
        On suppose que l'on dispose d'un tableau \lstinline[language=c]{int notes[N]} de \texttt{N} notes d'étudiants.
        On souhaite représenter ces notes sous la forme d'un histogramme horizontal, par exemple :

        \UPSTIquestion{Dans un premier temps, tenter de faire l'exercice sans aide. Si c'est trop difficile, lire les questions suivantes.}
        \UPSTIquestion{On suppose que les notes sont comprises entre 0 et 20. Créer un tableau \texttt{frequence[21]} initialisé à 0.}
        \UPSTIquestion{Pour chaque note du tableau \texttt{notes}, incrémenter la case correspondante.}
        \UPSTIquestion{Afficher ensuite un histogramme avec une étoile par étudiant (ex : 12 : ***).}
        \UPSTIquestion{Que faudrait-il modifier pour accepter des notes réelles (float) ?}
    \end{minipage}\hfill
    \begin{minipage}{.15\linewidth}
        \begin{tabular}{ll}
            0 &        \\
            1 & **     \\
            2 & *      \\
            3 & *      \\
            4 & ****   \\
            5 & **     \\
            6 & ***    \\
            7 & ****   \\
            8 & *****  \\
            9 & ****** \\
        \end{tabular}
    \end{minipage}

\end{UPSTIexercice}

\begin{UPSTIexercice}{Tableaux à deux dimensions}
    On souhaite manipuler une matrice $3 \times 3$.

    \UPSTIquestion{Déclarer un tableau \texttt{int M[3][3]}.}
    \UPSTIquestion{Écrire deux boucles imbriquées pour le remplir.}
    \UPSTIquestion{Afficher la matrice sous forme de grille.}
    \UPSTIquestion{Écrire une fonction \texttt{int somme\_diagonale(int M[3][3])}.}
\end{UPSTIexercice}