Cette partie explore l'utilisation de tableaux de structures, une combinaison puissante pour gérer des collections de données complexes.

\subsection{Premiers tableaux de structures}

\begin{UPSTIManipulation}{Polygone — Calculer le périmètre}

  \textbf{Contexte :} Un polygone est défini par une séquence de points. On veut calculer son périmètre.

  \textbf{Cahier des charges :}
  \begin{itemize}
    \item Utiliser la structure \texttt{Point} (x, y)
    \item Demander le nombre de sommets \textbf{n} ($3 \le n \le 20$)
    \item Lire les coordonnées de \textbf{n} points dans un tableau \texttt{Point sommets[20]}
    \item Calculer le périmètre : somme des distances entre points consécutifs
    \item Ne pas fermer le polygone (pas de distance entre dernier et premier point)
    \item Afficher le périmètre avec 2 décimales
  \end{itemize}

  \textbf{Exemple :}
  \begin{lstlisting}[style=console]
Nombre de sommets: 4
Point 0: (0, 0)
Point 1: (1, 0)
Point 2: (1, 1)
Point 3: (0, 1)
Perimetre: 3.00
  \end{lstlisting}

%   \tcblower
%   \begin{lstlisting}[language=bash]
% check50 IUT-GEII-Annecy/exercices/2025/info1/tp_structures/perimetre_polygone
%   \end{lstlisting}
\end{UPSTIManipulation}

\begin{UPSTIManipulation}{Trouver le point le plus proche de l'origine}

  \textbf{Contexte :} Parmi un ensemble de points, identifier celui qui est le plus proche de l'origine.

  \textbf{Cahier des charges :}
  \begin{itemize}
    \item Demander \textbf{n} ($1 \le n \le 50$) puis lire \textbf{n} points
    \item Trouver le point ayant la distance minimale à l'origine
    \item Afficher l'indice de ce point (0..n-1) et sa distance à l'origine (2 décimales)
    \item En cas d'égalité, afficher le premier trouvé
  \end{itemize}

  \textbf{Exemple :}
  \begin{lstlisting}[style=console]
Nombre de points: 3
Point 0: (5, 5)
Point 1: (1, 1)
Point 2: (2, 3)
Point le plus proche: indice 1, distance 1.41
  \end{lstlisting}

%   \tcblower
%   \begin{lstlisting}[language=bash]
% check50 IUT-GEII-Annecy/exercices/2025/info1/tp_structures/point_plus_proche
%   \end{lstlisting}
\end{UPSTIManipulation}

% \subsection{Structures plus complexes}

% \begin{UPSTIManipulation}{Gestion de capteurs — Température moyenne}

%   \textbf{Contexte :} Un système industriel dispose de plusieurs capteurs de température.

%   \textbf{Structure à utiliser :}
%   \begin{lstlisting}[language=C]
% typedef struct {
%     int id;              // Identifiant du capteur
%     float temperature;   // Temperature en degres Celsius
%     int actif;          // 1 si actif, 0 si inactif
% } Capteur;
%   \end{lstlisting}

%   \textbf{Cahier des charges :}
%   \begin{itemize}
%     \item Demander \textbf{n} ($1 \le n \le 30$) capteurs
%     \item Pour chaque capteur, lire : id, température, état (0 ou 1)
%     \item Calculer et afficher la \textbf{température moyenne} des capteurs \textbf{actifs uniquement}
%     \item Si aucun capteur actif, afficher \texttt{Aucun capteur actif}
%     \item Afficher la moyenne avec 2 décimales
%   \end{itemize}

%   \textbf{Exemple :}
%   \begin{lstlisting}[style=console]
% Nombre de capteurs: 3
% Capteur 0 - ID: 101, Temperature: 22.5, Actif (1/0): 1
% Capteur 1 - ID: 102, Temperature: 18.0, Actif (1/0): 0
% Capteur 2 - ID: 103, Temperature: 25.5, Actif (1/0): 1
% Temperature moyenne: 24.00
%   \end{lstlisting}

%   \tcblower
%   \begin{lstlisting}[language=bash]
% check50 IUT-GEII-Annecy/exercices/2025/info1/tp_structures/capteurs_temperature
%   \end{lstlisting}
% \end{UPSTIManipulation}

% \begin{UPSTIManipulation}{Capteurs — Identifier les alertes}
%   \textbf{} \texttt{tp\_structures/08\_capteurs\_alertes}

%   \textbf{Contexte :} Détecter les capteurs actifs dont la température dépasse un seuil.

%   \textbf{Cahier des charges :}
%   \begin{itemize}
%     \item Utiliser la même structure \texttt{Capteur}
%     \item Demander \textbf{n} capteurs et les lire
%     \item Demander un \textbf{seuil} de température
%     \item Afficher les \textbf{IDs} des capteurs actifs dont la température \textbf{dépasse strictement} le seuil
%     \item Si aucun, afficher \texttt{Aucune alerte}
%   \end{itemize}

%   \textbf{Exemple :}
%   \begin{lstlisting}[style=console]
% Nombre de capteurs: 4
% (Saisie des capteurs...)
% Seuil: 23.0
% Alertes: 101 103
%   \end{lstlisting}

%   \tcblower
%   \begin{lstlisting}[language=bash]
% check50 IUT-GEII-Annecy/exercices/2025/info1/tp_structures/capteurs_alertes
%   \end{lstlisting}
% \end{UPSTIManipulation}

\pagebreak
